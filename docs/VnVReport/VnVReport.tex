\documentclass[12pt, titlepage]{article}

%% Comments

\usepackage{color}

\newif\ifcomments\commentstrue

\ifcomments
\newcommand{\authornote}[3]{\textcolor{#1}{[#3 ---#2]}}
\newcommand{\todo}[1]{\textcolor{red}{[TODO: #1]}}
\else
\newcommand{\authornote}[3]{}
\newcommand{\todo}[1]{}
\fi

\newcommand{\wss}[1]{\authornote{blue}{SS}{#1}} 
\newcommand{\plt}[1]{\authornote{magenta}{TPLT}{#1}} %For explanation of the template
\newcommand{\an}[1]{\authornote{cyan}{Author}{#1}}

%% Common Parts
\usepackage{booktabs}
\usepackage{tabularx}
\usepackage{hyperref}
\usepackage[round]{natbib}
\usepackage{amsmath, mathtools}
\usepackage{amsfonts}
\usepackage{amssymb}
\usepackage{graphicx}
\usepackage{colortbl}
\usepackage{xr}
\usepackage{longtable}
\usepackage{xfrac}
\usepackage{float}
\usepackage{siunitx}
\usepackage{caption}
\usepackage{pdflscape}
\usepackage{afterpage}
\usepackage{tikz}
\usetikzlibrary{mindmap}
\usepackage{multirow}
\usepackage{fullpage}
%\usepackage{refcheck}

\newcommand{\famname}{MIA}
\newcommand{\progname}{MISEG}

% For easy change of table widths
\newcommand{\colZwidth}{1.0\textwidth}
\newcommand{\colAwidth}{0.13\textwidth}
\newcommand{\colBwidth}{0.82\textwidth}
\newcommand{\colCwidth}{0.1\textwidth}
\newcommand{\colDwidth}{0.05\textwidth}
\newcommand{\colEwidth}{0.8\textwidth}
\newcommand{\colFwidth}{0.17\textwidth}
\newcommand{\colGwidth}{0.5\textwidth}
\newcommand{\colHwidth}{0.28\textwidth}

% Used so that cross-references have a meaningful prefix
\newcounter{defnum} %Definition Number
\newcommand{\dthedefnum}{GD\thedefnum}
\newcommand{\dref}[1]{GD\ref{#1}}
\newcounter{datadefnum} %Datadefinition Number
\newcommand{\ddthedatadefnum}{DD\thedatadefnum}
\newcommand{\ddref}[1]{DD\ref{#1}}
\newcounter{theorynum} %Theory Number
\newcommand{\tthetheorynum}{T\thetheorynum}
\newcommand{\tref}[1]{T\ref{#1}}
\newcounter{tablenum} %Table Number
\newcommand{\tbthetablenum}{T\thetablenum}
\newcommand{\tbref}[1]{TB\ref{#1}}
\newcounter{assumpnum} %Assumption Number
\newcommand{\atheassumpnum}{P\theassumpnum}
\newcommand{\aref}[1]{A\ref{#1}}
\newcounter{goalnum} %Goal Number
\newcommand{\gthegoalnum}{P\thegoalnum}
\newcommand{\gsref}[1]{GS\ref{#1}}
\newcounter{instnum} %Instance Number
\newcommand{\itheinstnum}{IM\theinstnum}
\newcommand{\iref}[1]{IM\ref{#1}}
\newcounter{reqnum} %Requirement Number
\newcommand{\rthereqnum}{P\thereqnum}
\newcommand{\rref}[1]{R\ref{#1}}
\newcounter{lcnum} %Likely change number
\newcommand{\lthelcnum}{LC\thelcnum}
\newcommand{\lcref}[1]{LC\ref{#1}}

\hypersetup{
    bookmarks=true,         % show bookmarks bar?
    colorlinks=true,       % false: boxed links; true: colored links
    linkcolor=red,          % color of internal links (change box color with linkbordercolor)
    citecolor=green,        % color of links to bibliography
    filecolor=magenta,      % color of file links
    urlcolor=cyan           % color of external links
}


\begin{document}

\title{Test Report: System Verification and Validation Plan for \progname{}} 
\author{Ao Dong}
\date{\today}
	
\maketitle

\pagenumbering{roman}

\section{Revision History}

\begin{tabularx}{\textwidth}{p{3cm}p{2cm}X}
\toprule {\bf Date} & {\bf Version} & {\bf Notes}\\
\midrule
Dec 17 & 1.0 & Initial Draft\\
\bottomrule
\end{tabularx}

~\newpage

\section{Symbols, Abbreviations and Acronyms}

\renewcommand{\arraystretch}{1.2}
\begin{tabular}{l l} 
  \toprule		
  \textbf{symbol} & \textbf{description}\\
  \midrule 
  T & Test\\
  \bottomrule
\end{tabular}\\

For the other symbols, abbreviations and acronym, see SRS Documentation at
\url{https://github.com/Ao99/MIA/blob/master/docs/SRS/SRS.pdf}

\newpage

\tableofcontents

\listoftables %if appropriate

\listoffigures %if appropriate

\newpage

\pagenumbering{arabic}

This document is a report on the results of a testing suite for \progname{}.
Detailed descriptions of the tests executed can be found in SystVnVPlan
\cite{Dong2019SystVnV}.

\section{Functional Requirements Evaluation}

\section{Nonfunctional Requirements Evaluation}

\subsection{Usability}

\begin{longtable}{p{10cm} l l}
\toprule
Questions & Sets of Answers & Answers\\ \midrule
Are there installation instructions? & \{yes,no\} & yes\\
Are the installation instructions linear? & \{yes, no, N/A\} & yes\\
Is there something in place to automate the installation? & \{yes*, no\} & no\\
Is there a means given to validate the installation? & \{yes*, no\} & no\\
How many steps were involved in the installation? & $\mathbb{N}$ & 5\\
How many software packages need to be installed? & $\mathbb{N}$ & 3\\
Run uninstall, if available. Any obvious problems? & \{yes*, no, n/a\} & n/a\\
Overall Impression & \{1 .. 10\} & 6\\
\bottomrule
\caption{Installability Grade Sheet~\cite{SmithEtAl2018}}
\label{Tb_installability}
\end{longtable}

\subsection{Correctness and Verifiability}

\begin{longtable}{p{8cm} l p{5cm}}
\toprule
Questions & Sets of Answers & Answers\\ \midrule
Are external libraries used? & \{yes*, no, unclear\} & yes: Java library
dcm4che.\\
Does the community have confidence in this library? & \{yes, no, unclear\} &
yes\\
Any reference to the requirements specifications of the program? & \{yes*, no,
unclear\} & yes: SRS \cite{Dong2019SRS}\\
What tools or techniques are used to build confidence of correctness? & string
&JUnit, PMD, SystVnVPlan \cite{Dong2019SystVnV}, UnitVnVPlan
\cite{Dong2019UnitVnV}
\\
(I) If there is a getting started tutorial, is the output as expected? & \{yes,
no*, n/a\} & yes\\
Overall impression? & \{1 .. 10\} & 10\\
\bottomrule
\caption{Correctness and Verifiability Grade Sheet~\cite{SmithEtAl2018}}
\label{Tb_correctnessVerifiability}
\end{longtable}

\subsection{Robustness}

\begin{longtable}{p{10cm} l l}
\toprule
Questions & Sets of Answers & Answers\\
\midrule

(I) Does the software handle garbage input reasonably? & \{yes, no*\} & yes\\
(I) For any plain text input files, if all new lines are replaced with new
lines and carriage returns, will the software handle this gracefully? & \{yes,
no*,
n/a\} & n/a\\
Overall impression? & \{1 .. 10\} & 8\\
\bottomrule
\caption{Robustness Grade Sheet~\cite{SmithEtAl2018}}
\label{Tb_robustness}
\end{longtable}

\subsection{Usability}

\an{I didn't have a tester to do the new user test, so some results are faked.}
Percentages of improvements in indicators which are the less the better (such
as time and number of misoperations) are measured by the following equation:

percentage of improvement = $\frac{\text{first-time result} - \text{second-time
result}}{\text{first-time result}} \times 100\%$

Percentages of improvements in indicators which are the more the better (such
as success rate) are measured by the following equation:

percentage of improvement = $\frac{\text{second-time result} - \text{first-time
result}}{\text{first-time result}} \times 100\%$

\begin{longtable}{l p{6cm} l p{4cm}}
\hline
Test ID & Question/test detail & Set of Answers & Answers\\ \hline
\multirow{3}{*}{\begin{tabular}[c]{@{}l@{}}(I)Learnability:\\ new
users\end{tabular}} & Time to completion & Seconds & 60\\
 & Number of misoperations & $\mathbb{N}$ & 10\\
 & Success rate & Percentage & 80\% \\ \hline
\multirow{6}{*}{\begin{tabular}[c]{@{}l@{}}(I)Memorability\\ second-time
                  users\end{tabular}} & Time to completion & Seconds & 40\\
 & Percentage of improvement & Percentage & 33.3\%\\
 & Number of misoperations & $\mathbb{N}$ & 8\\
 & Percentage of improvement & Percentage & 20\%\\
 & Success rate & Percentage & 90\%\\
 & Percentage of improvement & Percentage & 12.5\%\\ \hline
\multirow{2}{*}{\begin{tabular}[c]{@{}l@{}}(I)Efficiency:\\ proficient
                  users\end{tabular}} & Time to completion & Seconds & 30\\
 & Number of misoperations & $\mathbb{N}$ & 2\\ \hline
\multirow{9}{*}{\begin{tabular}[c]{@{}l@{}}(I)Satisfaction:\\ every
user\end{tabular}} & Do the operations fit to human nature and your intuition?
&\{yes, no*\} & yes\\
 & Does it support your language? & \{yes, no*\} & yes\\
 & Can you understand the descriptions easily & \{yes, no*\} & yes\\
& Does it give a clear explanation when an error occurs? & \{yes, no*\} & yes\\
& Have you noticed any hot keys? & \{yes*, no\} & no\\
 & Do you think any hot key need to be added? & \{yes*, no\} & no\\
& Do you think undo or redo function is missing during any step? & \{yes*, no\}
& yes: there can be undo and redo for input functions
\\
& Do you think any other function for convenience need to be added? Such as
auto-fill, repeat and a record for all the steps. & \{yes*, no\} & yes: a
record of steps can be added\\
 & Overall satisfaction & \{1 .. 10\} & 8\\
\hline
\caption{Usability Grade Sheet}
\label{Tb_usability}
\end{longtable}

\subsection{Maintainability}

\begin{longtable}{p{7cm} p{4cm} l}
\hline Questions & Sets of Answers & Answers \\ \hline
Is there a history of multiple versions of the software? & \{yes, no, unclear\}
& no\\
Is there any information on how code is reviewed, or how to contribute? &
\{yes*, no\} & yes: SystVnVPlan \cite{Dong2019SystVnV}\\
Is there a change log? & \{yes, no\} & yes\\
What is the maintenance type? & \{corrective, adaptive, perfective, unclear\}&
corrective\\
What issue tracking tool is employed? & \{Trac, JIRA, Redmine, e-mail,
discussion board, SourceForge, Git, none, unclear\}& Git\\
Are the majority of identified bugs fixed? & \{yes, no*, unclear\} & yes\\
Which version control system is in use? & \{svn, cvs, git, github, unclear\} &
github\\
Is there evidence that maintainability was considered in the design? & \{yes*,
no\} & yes: SRS \cite{Dong2019SRS}\\
Are there code clones? & \{yes*, no, unclear\} & yes: on GitHub \\
Overall impression? & \{1 .. 10\} & 7\\ \hline
\caption{Maintainability Grade Sheet~\cite{SmithEtAl2018}}
\label{Tb_maintainability}
\end{longtable}

\subsection{}

\begin{longtable}{p{7cm} p{4cm} p{4cm}}
\toprule
Questions & Sets of Answers & Answers\\ \midrule

(I)What platforms is the software advertised to work on?
& \{Windows, Linux, macOS, Android, Other
OS\} & Windows, Linux, macOS\\
(I)Is there any compromise to functional or nonfunctional requirements by
running on this platform? &
\{yes*, no\} & no\\Are special steps taken in the source code to handle
portability?& \{yes*, no,
n/a\} & no\\
Is portability explicitly identified as NOT being important? & \{yes, no\} &
no\\
Convincing evidence that portability has been achieved? & \{yes*, no\} & yes:
tested on the platforms and Java has very good portability\\
Overall impression? & \{1 .. 10\} & 10\\ \bottomrule
\caption{Portability Grade Sheet~\cite{SmithEtAl2018}}
\label{Tb_portability}
\end{longtable}

\subsection{Understandability}

\begin{longtable}{p{7cm} p{3cm} p{5cm}}
\toprule
Questions & Set of Answers & Answers\\ \midrule
Consistent indentation and formatting style? & \{yes, no, n/a\} & yes\\
Explicit identification of a coding standard? & \{yes*, no, n/a\} & no\\
Are the code identifiers consistent, distinctive, and meaningful? & \{yes, no*,
n/a\} & yes\\
Are constants (other than 0 and 1) hard-coded into the program? & \{yes, no*,
n/a\} & no: there is a Constants module\\
Comments are clear, indicate what is being done, not how? & \{yes, no*, n/a\} &
yes
\\Is the name/URL of any algorithms used mentioned? & \{yes, no*, n/a\} & yes\\
Parameters are in the same order for all functions? & \{yes, no*, n/a\} & yes\\
Is code modularized? & \{yes, no*, n/a\} & yes\\
Descriptive names of source code files? & \{yes, no*, n/a\} & yes\\
Is a design document provided? & \{yes*, no, n/a\} & yes: MIS
\cite{Dong2019MIS}\\
Overall impression? & \{1 .. 10\} & 10\\ \bottomrule
\caption{Understandability Grade Sheet~\cite{SmithEtAl2018}}
\label{Tb_understandability}
\end{longtable}

\section{Comparison to Existing Implementation}	

This section will not be appropriate for every project.

\section{Unit Testing}

\section{Changes Due to Testing}

\section{Automated Testing}
		
\section{Trace to Requirements}
		
\section{Trace to Modules}		

\section{Code Coverage Metrics}

\newpage

\bibliographystyle{plainnat}

\bibliography{../../refs/References}

\end{document}
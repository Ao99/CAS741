\documentclass[12pt]{article}

%% Comments

\usepackage{color}

\newif\ifcomments\commentstrue

\ifcomments
\newcommand{\authornote}[3]{\textcolor{#1}{[#3 ---#2]}}
\newcommand{\todo}[1]{\textcolor{red}{[TODO: #1]}}
\else
\newcommand{\authornote}[3]{}
\newcommand{\todo}[1]{}
\fi

\newcommand{\wss}[1]{\authornote{blue}{SS}{#1}} 
\newcommand{\plt}[1]{\authornote{magenta}{TPLT}{#1}} %For explanation of the template
\newcommand{\an}[1]{\authornote{cyan}{Author}{#1}}

%% Common Parts
\usepackage{booktabs}
\usepackage{tabularx}
\usepackage{hyperref}
\usepackage[round]{natbib}
\usepackage{amsmath, mathtools}
\usepackage{amsfonts}
\usepackage{amssymb}
\usepackage{graphicx}
\usepackage{colortbl}
\usepackage{xr}
\usepackage{longtable}
\usepackage{xfrac}
\usepackage{float}
\usepackage{siunitx}
\usepackage{caption}
\usepackage{pdflscape}
\usepackage{afterpage}
\usepackage{tikz}
\usetikzlibrary{mindmap}
\usepackage{multirow}
\usepackage{fullpage}
%\usepackage{refcheck}

\newcommand{\famname}{MIA}
\newcommand{\progname}{MISEG}

% For easy change of table widths
\newcommand{\colZwidth}{1.0\textwidth}
\newcommand{\colAwidth}{0.13\textwidth}
\newcommand{\colBwidth}{0.82\textwidth}
\newcommand{\colCwidth}{0.1\textwidth}
\newcommand{\colDwidth}{0.05\textwidth}
\newcommand{\colEwidth}{0.8\textwidth}
\newcommand{\colFwidth}{0.17\textwidth}
\newcommand{\colGwidth}{0.5\textwidth}
\newcommand{\colHwidth}{0.28\textwidth}

% Used so that cross-references have a meaningful prefix
\newcounter{defnum} %Definition Number
\newcommand{\dthedefnum}{GD\thedefnum}
\newcommand{\dref}[1]{GD\ref{#1}}
\newcounter{datadefnum} %Datadefinition Number
\newcommand{\ddthedatadefnum}{DD\thedatadefnum}
\newcommand{\ddref}[1]{DD\ref{#1}}
\newcounter{theorynum} %Theory Number
\newcommand{\tthetheorynum}{T\thetheorynum}
\newcommand{\tref}[1]{T\ref{#1}}
\newcounter{tablenum} %Table Number
\newcommand{\tbthetablenum}{T\thetablenum}
\newcommand{\tbref}[1]{TB\ref{#1}}
\newcounter{assumpnum} %Assumption Number
\newcommand{\atheassumpnum}{P\theassumpnum}
\newcommand{\aref}[1]{A\ref{#1}}
\newcounter{goalnum} %Goal Number
\newcommand{\gthegoalnum}{P\thegoalnum}
\newcommand{\gsref}[1]{GS\ref{#1}}
\newcounter{instnum} %Instance Number
\newcommand{\itheinstnum}{IM\theinstnum}
\newcommand{\iref}[1]{IM\ref{#1}}
\newcounter{reqnum} %Requirement Number
\newcommand{\rthereqnum}{P\thereqnum}
\newcommand{\rref}[1]{R\ref{#1}}
\newcounter{lcnum} %Likely change number
\newcommand{\lthelcnum}{LC\thelcnum}
\newcommand{\lcref}[1]{LC\ref{#1}}

\hypersetup{
    bookmarks=true,         % show bookmarks bar?
    colorlinks=true,       % false: boxed links; true: colored links
    linkcolor=red,          % color of internal links (change box color with linkbordercolor)
    citecolor=green,        % color of links to bibliography
    filecolor=magenta,      % color of file links
    urlcolor=cyan           % color of external links
}


%tikstyle commands
% Define block styles
\tikzstyle{context} = [rectangle, rounded corners,  text width=2.5cm, minimum size=2.5cm,text centered, draw=black, fill=white!30]
\tikzstyle{strategy} = [trapezium, trapezium left angle=70, trapezium right angle=110, text width=3cm, minimum height=1cm, text centered, draw=black, fill=white!30]
\tikzstyle{subclaim} = [rectangle, text width=3cm, minimum height=1cm, text centered, draw=black, fill=white!30]
\tikzstyle{goal} = [rectangle, text width=5cm, minimum height=1cm, text centered, draw=black, fill=white!30]
\tikzstyle{assumption} = [ellipse, text width=3cm, minimum height=1cm, text centered, draw=black, fill=white!30]
\tikzstyle{evidence} = [circle, text width=2cm, minimum size=2cm, text centered, draw=black, fill=white!30]
\tikzstyle{arrow} = [thick,->,>=stealth]

\begin{document}

\title{Commonalities Analysis for Medical Imaging Applications} 
\author{Ao Dong}
\date{\today}

\maketitle

~\newpage

\pagenumbering{roman}

\section{Revision History}

\begin{tabularx}{\textwidth}{p{3cm}p{2cm}X}
\toprule {\bf Date} & {\bf Version} & {\bf Notes}\\
\midrule
Oct 11 & 1.0 & Initial draft\\
Oct 15 & 1.1 & Revised according to GitHub issues\\
Oct 17 & 1.2 & Added more nonfunctional requirements\\
Oct 18 & 1.3 & Revised input image format and functional requirements\\
Oct 21 & 1.4 & Revised TM, IM and nonfunctional requirements\\
\bottomrule
\end{tabularx}

~\newpage
	
\section{Reference Material}

This section records information for easy reference.

\subsection{Table of Units}

Throughout this document SI (Syst\`{e}me International d'Unit\'{e}s) is employed
as the unit system.  In addition to the basic units, several derived units are
used as described below.  For each unit, the symbol is given followed by a
description of the unit and the SI name.
~\newline

\renewcommand{\arraystretch}{1.2}
%\begin{table}[ht]
  \noindent \begin{tabular}{l l l} 
    \toprule		
    \textbf{symbol} & \textbf{unit} & \textbf{SI}\\
    \midrule 
    N/A\\
    \bottomrule
  \end{tabular}
  %	\caption{Provide a caption}
%\end{table}

\subsection{Table of Symbols}

The table that follows summarizes the symbols used in this document along with
their units.  The choice of symbols was made to be consistent with the heat
transfer literature and with existing documentation for solar water heating
systems.  The symbols are listed in alphabetical order.

\renewcommand{\arraystretch}{1.2}
%\noindent \begin{tabularx}{1.0\textwidth}{l l X}
\noindent \begin{longtable*}{l l p{12cm}} \toprule
\textbf{symbol} & \textbf{unit} & \textbf{description}\\
\midrule 
$a$ & N/A & dimension of spatial coordinates
\\
$b$ & N/A & dimension of feature values
\\
$C_{1}$ & N/A & the first class with pixels in $[0, k]$
\\
$C_{2}$ & N/A & the second class with pixels in $[k+1, L-1]$
\\
$f$ & N/A & function defining an input image
\\
$F$ & N/A & input medical image
\\
$g$ & N/A & function defining an output image
\\
$G$ & N/A & output segmentation image
\\
$h$ & N/A & function defining a mathematical image
\\
$H$ & N/A & 2D digital grayscale image
\\
$i$ & N/A & intensity value
\\
$k$ & N/A & threshold value in Otsu' Method
\\
$k_{1}$ & N/A & threshold value in Otsu' Method with multiple thresholds
\\
$k_{2}$ & N/A & threshold value in Otsu' Method with multiple thresholds
\\
$k^{\star}$ & N/A & optimal  threshold  value found by Otsu' Method
\\
$k^{\star}_{1}$ & N/A & optimal  threshold  value found by Otsu' Method with multiple thresholds
\\
$k^{\star}_{2}$ & N/A & optimal  threshold  value found by Otsu' Method with multiple thresholds
\\
$L$ & N/A & number  of  the  discrete  levels  of  the  feature value
\\
$m_{1}$ & N/A & mean intensity of the pixels in $C_{1}$
\\
$m_{2}$ & N/A & mean intensity of the pixels in $C_{2}$
\\
$m_{3}$ & N/A & mean intensity of the pixels in the third class
\\
$m_{G}$ & N/A & mean global intensity
\\
$n_{i}$ & N/A & number of pixels of intensity $i$
\\
$\mathbb{N}$ & N/A & set of all natural numbers
\\
$p_{i}$ & N/A & normalized histogram
\\
$P_{1}$ & N/A & probability of the first class $C_{1}$
\\
$P_{2}$ & N/A & probability of the second class $C_{2}$
\\
$P_{3}$ & N/A & probability of the third class
\\
$\mathbb{R}$ & N/A & set of all real numbers
\\
$\sigma_{B}$ & N/A & between-class variance
\\
$T_{h}$ & N/A & threshold value for segmentation
\\
$x$ & N/A & x-axial coordinate of a image
\\
$y$ & N/A & y-axial coordinate of a image
\\
$X$ & N/A & x-axial pixel length of an image
\\
$Y$ & N/A & y-axial pixel length of an image
\\
$\mathbb{Z}$ & N/A & set of all integers
\\ 
\bottomrule
\end{longtable*}

\subsection{Abbreviations and Acronyms}

\renewcommand{\arraystretch}{1.2}
\begin{tabular}{l l} 
  \toprule		
  \textbf{symbol} & \textbf{description}\\
  \midrule 
  2D & Two-Dimensional\\
  3D & Three-Dimensional\\
  A & Assumption\\
  CA & Commonalities Analysis\\
  DD & Data Definition\\
  DICOM & Digital Imaging and Communications in Medicine\\
  GD & General Definition\\
  GS & Goal Statement\\
  IM & Instance Model\\
  LC & Likely Change\\
  MG & Module Guide\\
  \famname & Medical Image Applications\\
  MIS & Module Interface Specification\\
  PS & Physical System Description\\
  R & Requirement\\
  T & Theoretical Model\\
  VTK & the Visualization Toolkit\\
  VnV & Verification and Validation\\
  \bottomrule
\end{tabular}\\

\newpage

\tableofcontents

~\newpage

\pagenumbering{arabic}

\section{Introduction}

Medical imaging technologies in medical diagnoses have become essential and effective tools for professionals in medicine. Computers are also widely used to generate, process, and visualize medical images. There is a big family of computer software developed to serve these purposes.

The following sections provide an overview of the Commonalities Analysis (CA) for the family of Medical Imaging Applications (\famname). This section explains the purpose of this document, the scope of the family, the characteristics of the intended reader, and the organization of the document. 

\subsection{Purpose of Document}

The major purpose of this document is to describe some commonalities of \famname. Goals, assumptions, theoretical models, definitions, and other model derivation information are specified, allowing the reader to fully understand and verify the purpose and scientific basis of \famname. This CA will remain abstract, describing what problem is being solved, but not how to solve it.

This document will be used as a starting point for subsequent development phases, including writing the design specification and the software verification and validation plan. The design document will show how the requirements are to be realized, including decisions
on the numerical algorithms and programming environment. The verification and validation plan will show the steps that will be used to increase confidence in the software documentation and the implementation. Although the CA fits in a series of documents that follow the so-called waterfall model, the actual development process is not constrained in any way. Even when the waterfall model is not followed, as \cite{ParnasAndClements1986} point out, the most logical way to present the documentation is still to “fake” a rational design process.

\subsection{Scope of the Family} 

According to \cite{Bankman2000}, \famname{} deal with 6 different basic problems, while \cite{Angenent2006} pointed out that 4 fundamental problems are solved by \famname. While both mentioned Segmentation, Registration and Visualization of medical images, Bankman also included Enhancement, Quantification and a section covering some other functions~\cite{Bankman2000}. On the other hand, Angenent's team included Simulation~\cite{Angenent2006}. According to \cite{wiki:001}, \famname{} have major functions in categories such as Segmentation, Registration, Visualization (including the basic display, reformatted views and 3D volume rendering), Statistical Analysis, Image-based Physiological Modelling, etc. As \cite{Kim2011} describe, the general steps of medical image analysis after obtaining digital data include Enhancement, Segmentation, Feature Extraction, Classification and Interpretation.

The major functions of \famname{} can be divided into several sections and sub sections as shown in Figure \ref{fg_miafunctions}.

\begin{center}
\begin{tikzpicture}[mindmap, grow cyclic, every node/.style=concept, concept color=orange!40, 
	level 1/.append style={level distance=5cm,sibling angle=90},
	level 2/.append style={level distance=3cm,sibling angle=90},
	level 3/.style={level distance=3cm,sibling angle=60}]
\node{\famname}
	child[concept color=teal!40] { node {Enhancement}}
	child[concept color=teal!40] { node {Analysis}
    	child[concept color=blue!30] { node {Registration}}
	    child[concept color=purple!50,] { node {Segmentation}}
	    child[concept color=blue!30] { node {Statistical Analysis}
	        child { node {Feature Extraction}}
	        child { node {Classification}}
	        child { node {Interpretation}}
	}}
	child[concept color=teal!40] { node {Simulation/\\Modeling}}
    child[concept color=teal!40] { node {Visualization}
	    child[concept color=blue!30] { node {2D Display}}
	    child[concept color=blue!30] { node {3D Rendering}}
	    child[concept color=blue!30] { node {Reformatted Views}}
    	}
;
\end{tikzpicture}
\captionof{figure}{Major functions of \famname}
\label{fg_miafunctions}
\end{center}

In this project, the scope of the family is limited to the software with the Segmentation functions.

\subsection{Characteristics of Intended Reader} 

Reviewers of this documentation should have an understanding of functions, sets and binary numbers in discrete math from level 1 or 2 computer science and probability from level 1 and 2 calculus.

The users of \famname{} can have a lower level of expertise, as explained in Section \ref{sec_UserCharacteristics}.


\subsection{Organization of Document}

The organization of this document follows the template for an CA for scientific computing software proposed by \cite{Parnas1972} and \cite{ParnasAndClements1986}. The presentation follows the standard pattern of presenting goals, theories, definitions, and assumptions. For readers that would like a more bottom up approach, they can start reading the instance
models in Section \ref{sec_instance} and trace back to find any additional information they require.
The goal statements (Section \ref{sec_goalstatements}) are refined to the theoretical models and the theoretical models (Section \ref{sec_theoretical}) to the instance models (Section \ref{sec_instance}). The instance models to be solved are referred to as \iref{IM_otsufindk}, \iref{IM_globaloutput}, \iref{IM_multifindk}, \iref{IM_multioutput}.

\section{General System Description}

This section identifies the interfaces between the system and its environment,
describes the potential user characteristics and lists the potential system
constraints.

\subsection{Potential System Contexts}

Figure \ref{fg_syscontext} shows the system context. A circle represents an external entity outside the software, the user in this case. A rectangle represents the software system itself (\famname).
Arrows are used to show the data flow between the system and its environment.
\famname{} are mostly self-contained. The only external interaction is through the user interface. The responsibilities of the user and the system are as follows:

\begin{itemize}
\item User Responsibilities:
\begin{itemize}
\item Provide the input data to the system
\item Given two or more options by the system, decide to use which calculation method
\end{itemize}
\item \famname{} Responsibilities:
\begin{itemize}
\item Detect data type mismatch, such as a text file instead of a image file
\item Determine if the inputs satisfy the required mathematical and software constraints
\item Calculate the required outputs
\end{itemize}
\end{itemize}

\begin{center}
\begin{tikzpicture}
    \node (n1) [evidence] {User};
    \node (n2) [context ,right of= n1, node distance = 2.1in] {\famname};
    \node (n3) [evidence, right of= n2, node distance = 4.4in] {User};
    \draw [arrow] (n1.0)  -- node[above]{Medical Image}  (n2.180);
    \draw [arrow] (n2.0)  -- node[above]{Optimal threshold values, segmentation image}  (n3.180);
\end{tikzpicture}
\captionof{figure}{System Context}
\label{fg_syscontext}
\end{center}

\subsection{Potential User Characteristics} \label{sec_UserCharacteristics}

The end user of \famname{} should have an understanding of undergraduate Level
1 Calculus and Physics.

\subsection{Potential System Constraints}

There are no system constraints.

\section{Commonalities}
This section first presents the background overview, which gives a high-level view of the
problem to be solved. This is followed by the solution characteristics specification, which
presents the assumptions, theories, definitions and finally the instance models.

\subsection{Background Overview} \label{sec_Background}
Segmentation, separation of structures of interest from the background and from each other~\cite{Bankman2000}. Image segmentation is the process of partitioning an image into different meaningful segments. In medical imaging, these segments often correspond to different tissue classes, organs, pathologies, or other biologically relevant structures~\cite{Forouzanfar2010}. Image segmentation is one of the most interesting and challenging problems in computer vision generally and medical imaging applications specifically~\cite{Elnakib2011}.

Regarding the different medical image segmentation methods, \cite{Withey2007} suggested that they can be divided into 3 generations from low-level to high-level technologies as shown in table \ref{fg_segmethods}.

This document is focusing on the Intensity Threshold method.

\begin{center}
\begin{table}[h]
\begin{tabular}{|c|l|l|l|}
\hline
\multirow{2}{*}{\textbf{Generation}} & \multicolumn{3}{c|}{\textbf{Category}} \\ \cline{2-4} 
 & \multicolumn{1}{c|}{\textbf{Region-based}} & \multicolumn{1}{c|}{\textbf{\begin{tabular}[c]{@{}c@{}}Boundary\\ Following\end{tabular}}} & \multicolumn{1}{c|}{\textbf{\begin{tabular}[c]{@{}c@{}}Pixel\\ Classification\end{tabular}}} \\ \hline
\textbf{$1^{st}$} & • Region growing & \begin{tabular}[c]{@{}l@{}}• Edge tracing\\ (heuristic)\end{tabular} & • Intensity threshold \\ \hline
\textbf{$2^{nd}$} & \begin{tabular}[c]{@{}l@{}}• Deformable models\\ • Graph search\end{tabular} & \begin{tabular}[c]{@{}l@{}}• Minimal path\\ • Target tracking\\ • Graph search\\ • Neural networks\\ • Multiresolution\end{tabular} & \begin{tabular}[c]{@{}l@{}}• Statistical pattern recognition\\ • C-means clustering\\ • Neural networks\\ • Multiresolution\end{tabular} \\ \hline
\textbf{$3^{rd}$} & \begin{tabular}[c]{@{}l@{}}• Shape models\\ • Appearance models\\ • Rule-based\\ • Coupled surfaces\end{tabular} &  & \begin{tabular}[c]{@{}l@{}}• Atlas-based\\ • Rule-based\end{tabular} \\ \hline
\end{tabular}
\caption{Segmentation Methods~\cite{Withey2007}}
\label{fg_segmethods}
\end{table}
\end{center}

\subsection{Terminology and  Definitions}

This subsection provides a list of terms that are used in the subsequent
sections and their meaning, with the purpose of reducing ambiguity and making it
easier to correctly understand the requirements:

\begin{itemize}

\item Image: in Mathematics, an image is defined as a function $h : \mathbb{R}^{a} \rightarrow \mathbb{R}^{b}$.
\item Digital image: When the spatial coordinates and the function value are finite and discrete, the image is called digital, shown as $h : \mathbb{Z}^{a} \rightarrow \mathbb{Z}^{b}$.
\item Grayscale image: In digital photography, computer-generated imagery, and colorimetry, a grayscale or greyscale image is one in which the value of each pixel is a single sample representing only an amount of light, that is, it carries only intensity information.
\item 2D Digital Image: the computer-based generation of digital images - mostly from two-dimensional models (such as 2D geometric models, text, and digital images) and by techniques specific to them.
\item Medical Image: visual representations of the interior of a body for clinical analysis and medical intervention, as well as visual representation of the function of some organs or tissues (physiology).
\item Grayscale Intensity: represents gray levels, where the intensity 0 usually represents black and the intensity 255 usually represents full intensity, or white.
\item Shades of Gray: variations of gray or grey include achromatic grayscale shades, which lie exactly between white and black, and nearby colors with low colorfulness.
\item Coordinates: in geometry, a coordinate system is a system that uses one or more numbers, or coordinates, to uniquely determine the position of the points or other geometric elements on a manifold such as Euclidean space.
\item Histogram: a histogram is an accurate representation of the distribution of numerical data. It is an estimate of the probability distribution of a continuous variable.
\item Pixel: in digital imaging, a pixel, pel, or picture element is a physical point in a raster image, or the smallest addressable element in an all points addressable display device; so it is the smallest controllable element of a picture represented on the screen.
\item VTK: the Visualization Toolkit is an open-source, freely available software system for 3D computer graphics, modeling, image processing, volume rendering, scientific visualization, and 2D plotting.
\item DICOM: Digital Imaging and Communications in Medicine (DICOM) is the standard for the communication and management of medical imaging information and related data.
\end{itemize}

\subsection{Data Definitions} \label{sec_datadef}

This section collects and defines all the data needed to build the instance models. The dimension of each quantity is also given.

~\newline

\noindent
\begin{minipage}{\textwidth}
\renewcommand*{\arraystretch}{1.5}
\begin{tabular}{| p{\colAwidth} | p{\colBwidth}|}
\hline
\rowcolor[gray]{0.9}
Number& DD\refstepcounter{datadefnum}\thedatadefnum \label{DD_digitalimage}\\
\hline
Label& \bf Digital Image\\
\hline
Symbol & N/A\\
\hline
% Units& $Mt^{-3}$\\
% \hline
  SI Units & N/A\\
  \hline
  Equation & $h : \mathbb{Z}^{a} \rightarrow \mathbb{Z}^{b}$\\
  \hline
  Description & 
    In Mathematics, an image is defined as a function $h : \mathbb{R}^{a} \rightarrow \mathbb{R}^{b}$. Usually, $a = 2$ and, in the simplest case, $b = 1$. When the spatial coordinates and the function value are finite and discrete, the image is called digital, shown as $h : \mathbb{Z}^{a} \rightarrow \mathbb{Z}^{b}$.      
  \\
  \hline
  Sources& \cite{Ferrari2018a}\\
  \hline
  Ref.\ By & \ddref{DD_2Dimage}\\
  \hline
\end{tabular}
\end{minipage}\\

~\newline

\noindent
\begin{minipage}{\textwidth}
\renewcommand*{\arraystretch}{1.5}
\begin{tabular}{| p{\colAwidth} | p{\colBwidth}|}
\hline
\rowcolor[gray]{0.9}
Number& DD\refstepcounter{datadefnum}\thedatadefnum \label{DD_2Dimage}\\
\hline
Label& \bf 2D Digital Grayscale Image\\
\hline
Symbol & $H$\\
\hline
% Units& $Mt^{-3}$\\
% \hline
  SI Units & N/A\\
  \hline
  Equation & $H_{X \times Y} = [h(x,y)]_{X \times Y}$\\
  \hline
  Description & 
    In this project, we only consider 2D grayscale images, as stated in \aref{A_2Dgrayscale}. The images can be defined as $h : \mathbb{Z}^{2} \rightarrow \mathbb{Z}$, or the above equation. $X \times Y$ is the size of the image. $(x,y)$ denotes the 2D spatial coordinates, where $x \in [0,X-1]$ and $y \in [0,Y-1]$
  \\
  \hline
  Sources& \cite{Pal1993}\\
  \hline
  Ref.\ By & \ddref{DD_inoutimage}\\
  \hline
\end{tabular}
\end{minipage}\\

~\newline

\noindent
\begin{minipage}{\textwidth}
\renewcommand*{\arraystretch}{1.5}
\begin{tabular}{| p{\colAwidth} | p{\colBwidth}|}
\hline
\rowcolor[gray]{0.9}
Number& DD\refstepcounter{datadefnum}\thedatadefnum \label{DD_inoutimage}\\
\hline
Label& \bf Input and Output Medical Image\\
\hline
Symbol & $F$ and $G$\\
\hline
% Units& $Mt^{-3}$\\
% \hline
  SI Units & N/A\\
  \hline
  Equation & $F_{X \times Y} = [f(x,y)]_{X \times Y}$ and $G_{X \times Y} = [g(x,y)]_{X \times Y}$\\
  \hline
  Description & 
    F and G denote the input and output medical image respectively, which are both 2D Digital Grayscale. $X \times Y$ is the size of the image. $(x,y)$ denotes the 2D spatial coordinates, where $x \in [0,X-1]$ and $y \in [0,Y-1]$
  \\
  \hline
  Sources& \cite{Pal1993}\\
  \hline
  Ref.\ By & \tref{T_globalthres} \tref{T_multithres} \iref{IM_otsufindk}\\
  \hline
\end{tabular}
\end{minipage}\\

~\newline

\noindent
\begin{minipage}{\textwidth}
\renewcommand*{\arraystretch}{1.5}
\begin{tabular}{| p{\colAwidth} | p{\colBwidth}|}
\hline
\rowcolor[gray]{0.9}
Number& DD\refstepcounter{datadefnum}\thedatadefnum \label{DD_featurevalue}\\
\hline
Label& \bf Input and Output Image Feature Value\\
\hline
Symbol & $f$ and $g$\\
\hline
% Units& $Mt^{-3}$\\
% \hline
  SI Units & N/A\\
  \hline
  Equation & $f(x,y), g(x,y) \in \{0,1,...,L-1\}$\\
  \hline
  Description & 
    $(x,y)$ denotes the 2D spatial coordinates and $f(x,y)$, $g(x,y)$ the feature values at $(x,y)$ of the input and output image respectively. Depending on the type of image, the feature value could be light intensity, depth, intensity of radio wave or temperature. $\{0,1,...,L-1\}$ is the set of discrete levels of the feature value, and $L$ is the number of the levels. In this project, we refer to $f(x,y)$ and $g(x,y)$ as gray intensity value (or intensity) at $(x,y)$.
  \\
  \hline
  Sources& \cite{Pal1993}\\
  \hline
  Ref.\ By & \ddref{DD_inoutimage} \ddref{DD_betweenvariance} \tref{T_globalthres} \tref{T_multithres} \iref{IM_otsufindk}\\
  \hline
\end{tabular}
\end{minipage}\\

~\newline

\noindent
\begin{minipage}{\textwidth}
\renewcommand*{\arraystretch}{1.5}
\begin{tabular}{| p{\colAwidth} | p{\colBwidth}|}
\hline
\rowcolor[gray]{0.9}
Number& DD\refstepcounter{datadefnum}\thedatadefnum \label{DD_numberofshadesgray}\\
\hline
Label& \bf Number of the shades of gray\\
\hline
Symbol & $L$\\
\hline
% Units& $Mt^{-3}$\\
% \hline
  SI Units & N/A\\
  \hline
  Equation & $L \in \{256, 4096, 65536\}$\\
  \hline
    Description & 
    $L$ is the number of the shades of gray, also referred as number of intensity levels, where $L \in \{2^{8}, 2^{12}, 2^{16}$, as stated in \aref{A_8bitinteger}.
  \\
  \hline
  Sources& \url{https://homepages.inf.ed.ac.uk/rbf/HIPR2/value.htm}\\
  \hline
  Ref.\ By & \ddref{DD_featurevalue} \ddref{DD_thresvalue} \ddref{DD_betweenvariance} \tref{T_globalthres} \tref{T_multithres} \tref{T_otsu} \iref{IM_otsufindk}\\
  \hline
\end{tabular}
\end{minipage}\\

~\newline

\noindent
\begin{minipage}{\textwidth}
\renewcommand*{\arraystretch}{1.5}
\begin{tabular}{| p{\colAwidth} | p{\colBwidth}|}
\hline
\rowcolor[gray]{0.9}
Number& DD\refstepcounter{datadefnum}\thedatadefnum \label{DD_thresvalue}\\
\hline
Label& \bf Threshold Value\\
\hline
Symbol & $T_{h}$\\
\hline
% Units& $Mt^{-3}$\\
% \hline
  SI Units & N/A\\
  \hline
  Equation & $T_{h} \in \{1,...,L-2\}$\\
  \hline
  Description & 
    In this project, we refer to $f(x,y)$ and $g(x,y)$ as gray intensity value (or intensity) at $(x,y)$. Then, $T_{h} \in [1,254]$.
  \\
  \hline
  Sources& \cite{Ferrari2018b}\\
  \hline
  Ref.\ By & \ddref{DD_betweenvariance} \tref{T_globalthres} \tref{T_multithres} \iref{IM_otsufindk}\\
  \hline
\end{tabular}
\end{minipage}\\

~\newline

\noindent
\begin{minipage}{\textwidth}
\renewcommand*{\arraystretch}{1.5}
\begin{tabular}{| p{\colAwidth} | p{\colBwidth}|}
\hline
\rowcolor[gray]{0.9}
Number& DD\refstepcounter{datadefnum}\thedatadefnum \label{DD_betweenvariance}\\
\hline
Label& \bf Between-class Variance\\
\hline
Symbol & $\sigma^{2}_{B}$\\
\hline
% Units& $Mt^{-3}$\\
% \hline
  SI Units & N/A\\
  \hline
  Equation & $\sigma^{2}_{B} = P_{1}(m_{1} - m_{G})^{2} + P_{2}(m_{2} - m_{G})^{2}$\\
  \hline
  Description &
    $n_{i}$ is the number of pixels of intensity $i$, where $i \in \{0,1,...,L-1\}.$
    
    $p_{i}$ is the normalized histogram, where
    
    $p_{i} = \frac{n_{i}}{\sum_{i=1}^{L-1} n_{i}}$.
    
    Using $k, 0 < k < L - 1$, as threshold, there are two classes: $C_{1}$ (pixels in $[0, k]$) and $C_{2}$ (pixels in $[k + 1, L - 1]$).

    $P_{1}$ is the probability of the class $C_{1}$, where 
    
    $P_{1} = P(C_{1}) = \sum_{i=1}^{k} p_{i}$.
    
    $P_{2}$ is the probability of the class $C_{2}$, where 
    
    $P_{2} = P(C_{2}) = \sum_{i=k+1}^{L-1} p_{i} = 1 - P_{1}$.
    
    $m_{1}$ is the mean intensity of the pixels in $C_{1}$, where
    
    $m_{1} = \sum_{i=1}^{k} i \cdot P(i | C_{1}) = \sum_{i=1}^{k} i \cdot \frac{P(C_{1}|i)P(i)}{P(C_{1})} = \frac{1}{P_{1}}\sum_{i=1}^{k} i \cdot p_{i}$, since $P(C_{1}|i)=1, P(i)=p_{i}$, and $P(C_{1})=P_{1}$.
    
    $m_{2}$ is the mean intensity of the pixels in $C_{2}$, similarly
    
    $m_{2} = \frac{1}{P_{2}}\sum_{i=k}^{L-1} i \cdot p_{i}$.
    
    $m_{G}$ is the mean global intensity, where
    
    $m_{G} = \sum_{i=1}^{L-1} i \cdot p_{i}$.
  \\
  \hline
  Sources& \cite{Ferrari2018b}\\
  \hline
  Ref.\ By & \tref{T_otsu} \iref{IM_otsufindk} \iref{IM_multifindk}\\
  \hline
\end{tabular}
\end{minipage}\\

~\newline

\subsection{Goal Statements}
\label{sec_goalstatements}
\noindent Given the medical images as inputs, the goal statements are:

\begin{itemize}

\item[GS\refstepcounter{goalnum}\thegoalnum \label{GS_calk}:]
Calculate and display the optimal threshold value $k^{\star}$ with Otsu’s Method.

\item[GS\refstepcounter{goalnum}\thegoalnum \label{GS_outputimage}:]
Output the processed images representing the segmentation results with one threshold.

\item[GS\refstepcounter{goalnum}\thegoalnum \label{GS_multicalk}:]
Calculate and display multiple optimal threshold values $k^{\star}_{1}$ and $k^{\star}_{2}$ with Otsu’s Method.

\item[GS\refstepcounter{goalnum}\thegoalnum \label{GS_multioutputimage}:]
Output the processed images representing the segmentation results with multiple thresholds.

\end{itemize}

\subsection{Theoretical Models} \label{sec_theoretical}

This section focuses on the general equations and laws that \famname{} is based on.

~\newline

\noindent
\begin{minipage}{\textwidth}
\renewcommand*{\arraystretch}{1.5}
\begin{tabular}{| p{\colAwidth} | p{\colBwidth}|}
  \hline
  \rowcolor[gray]{0.9}
  Number& T\refstepcounter{theorynum}\thetheorynum \label{T_mathimage}\\
  \hline
  Label&\bf Mathematical Image\\
  \hline
  Equation&  $h : \mathbb{R}^{a} \rightarrow \mathbb{R}^{b}$\\
  \hline
  Description & 
    In Mathematics, an image is defined as a function $h : \mathbb{R}^{a} \rightarrow \mathbb{R}^{b}$, where $\mathbb{R}^{a}$ denotes the set of a-dimensional spatial coordinates and $\mathbb{R}^{b}$ the set of b-dimensional feature values.\\ 
  \hline
  Source &  \cite{Ferrari2018a}\\
  % The above web link should be replaced with a proper citation to a publication
  \hline
  Ref.\ By & \ddref{DD_digitalimage}\\
  \hline
\end{tabular}
\end{minipage}\\

~\newline

\noindent
\begin{minipage}{\textwidth}
\renewcommand*{\arraystretch}{1.5}
\begin{tabular}{| p{\colAwidth} | p{\colBwidth}|}
  \hline
  \rowcolor[gray]{0.9}
  Number& T\refstepcounter{theorynum}\thetheorynum \label{T_globalthres}\\
  \hline
  Label&\bf Single Global Threshold Method\\
  \hline
  Equation&  $g(x,y)=\left\{
\begin{aligned}
&L-1,\ &if\ f(x,y)>T_{h} \\
&0,\ &if\ f(x,y)\leq T_{h}
\end{aligned}
\right.$\\
  \hline
  Description & 
    $f(x,y)$ and $g(x,y)$ are the feature values at $(x,y)$ of the input and output image respectively. $L$ is the number of the discrete levels of the feature value. $T_{h}$ is the threshold value.\\
  \hline
  Source &  \cite{Ferrari2018b}\\
  % The above web link should be replaced with a proper citation to a publication
  \hline
  Ref.\ By & \ddref{DD_betweenvariance} \tref{T_multithres} \iref{IM_globaloutput}\\
  \hline
\end{tabular}
\end{minipage}\\

~\newline

\noindent
\begin{minipage}{\textwidth}
\renewcommand*{\arraystretch}{1.5}
\begin{tabular}{| p{\colAwidth} | p{\colBwidth}|}
  \hline
  \rowcolor[gray]{0.9}
  Number& T\refstepcounter{theorynum}\thetheorynum \label{T_multithres}\\
  \hline
  Label&\bf Multiple Global Thresholds Method\\
  \hline
  Equation&  $g(x,y)=\left\{
\begin{aligned}
&L-1,\ &if\ f(x,y) > T_{h2} \\
&L/2,\ &if\ T_{h1} < f(x,y) \leq T_{h2}\\
&0,\ &if\ f(x,y) \leq T_{h1}
\end{aligned}
\right.$\\
  \hline
  Description & 
    $f(x,y)$ and $g(x,y)$ are the feature values at $(x,y)$ of the input and output image respectively. $L$ is the number of the discrete levels of the feature value. $T_{h1}$ and $T_{h2}$ are the threshold values.\\
  \hline
  Source &  \cite{Ferrari2018b}\\
  % The above web link should be replaced with a proper citation to a publication
  \hline
  Ref.\ By & \iref{IM_multifindk}\\
  \hline
\end{tabular}
\end{minipage}\\

~\newline

\noindent
\begin{minipage}{\textwidth}
\renewcommand*{\arraystretch}{1.5}
\begin{tabular}{| p{\colAwidth} | p{\colBwidth}|}
  \hline
  \rowcolor[gray]{0.9}
  Number& T\refstepcounter{theorynum}\thetheorynum \label{T_otsu}\\
  \hline
  Label&\bf Otsu's Method\\
  \hline
  Equation&  $\sigma^{2}_{B}(k^{\star}) = \underset{0<k<L-1}{max}\sigma^{2}_{B}(k)$\\
  \hline
  Description & 
    Otsu’s method is aimed in finding the optimal value for the global threshold $T$. The optimal threshold value, $k^{\star}$, satisfies the above equation, where using $k$ as threshold to divide all pixels of the image into two classes: $C_{1}$ (pixels in $[0, k]$) and $C_{2}$ (pixels in $[k + 1, L - 1]$). $\sigma_{B}$ is the between-class variance.\\ 
  \hline
  Source &  \cite{Ferrari2018b}\\
  % The above web link should be replaced with a proper citation to a publication
  \hline
  Ref.\ By & \iref{IM_otsufindk} \iref{IM_multifindk}\\
  \hline
\end{tabular}
\end{minipage}\\

~\newline

\subsection{Instance Models} \label{sec_instance}    

This section transforms the problem defined in Section~\ref{sec_Background} into
one which is expressed in mathematical terms. It uses concrete symbols defined 
in Section~\ref{sec_datadef} to replace the abstract symbols in the models identified in Sections~\ref{sec_theoretical}.

Medical image segmentation by Threshold Method can be solved by \tref{T_globalthres}, \tref{T_otsu}, \ddref{DD_inoutimage}, \ddref{DD_featurevalue}, \ddref{DD_thresvalue}

~\newline

\noindent
\begin{minipage}{\textwidth}
\renewcommand*{\arraystretch}{1.5}
\begin{tabular}{| p{\colAwidth} | p{\colBwidth}|}
  \hline
  \rowcolor[gray]{0.9}
  Number& IM\refstepcounter{instnum}\theinstnum \label{IM_otsufindk}\\
  \hline
  Label& \bf Otsu's Method to find the single optimal threshold value $k^{\star}$\\
  \hline
  Input& $F_{X \times Y}$ from \ddref{DD_inoutimage}, $f(x,y)$ from \ddref{DD_featurevalue}, $L$ from \ddref{DD_numberofshadesgray}, $\sigma_{B}$, k, $P_{1}$, $P_{2}$, $m_{1}$, $m_{2}$, $m_{G}$ from \ddref{DD_betweenvariance}\\
  \hline
  Output& $k^{\star}$, such that\\
  & $\sigma^{2}_{B}(k^{\star}) = \underset{0<k<L-1}{max}\sigma^{2}_{B}(k)$\\
  \hline
  Description&
        Since $P_{1}m_{1} + P_{2}m_{2} = m_{G}$ and $P_{1} + P_{2} = 1$,
        
        $\sigma^{2}_{B} = P_{1}(m_{1} - m_{G})^{2} + P_{2}(m_{2} - m_{G})^{2} = P_{1}P_{2}(m_{1} - m_{2})^{2}$
        
        Hence, for each value of $k$, $\sigma_{B}$ can be computed, where 
        
        $\sigma^{2}_{B}(k) = P_{1}(k)P_{2}(k)(m_{1}(k) - m_{2}(k))^{2}$
  \\
  \hline
  Sources& \cite{Ferrari2018b} \\
  \hline
  Ref.\ By & \iref{IM_globaloutput}\\
  \hline
\end{tabular}
\end{minipage}\\

~\newline

\noindent
\begin{minipage}{\textwidth}
\renewcommand*{\arraystretch}{1.5}
\begin{tabular}{| p{\colAwidth} | p{\colBwidth}|}
  \hline
  \rowcolor[gray]{0.9}
  Number& IM\refstepcounter{instnum}\theinstnum \label{IM_globaloutput}\\
  \hline
  Label& \bf Use single global threshold to output segmentation image\\
  \hline
  Input& $F_{X \times Y}$ from \ddref{DD_inoutimage}, $f(x,y)$ from \ddref{DD_featurevalue}, $L$ from \ddref{DD_numberofshadesgray}, $k^{\star}$ from \iref{IM_otsufindk}\\
  \hline
  Output& $G_{X \times Y}$, such that for pixel at each $(x,y)$,
  
  $g(x,y)=\left\{
\begin{aligned}
&255,\ &if\ f(x,y)>k^{\star} \\
&0,\ &if\ f(x,y)\leq k^{\star}
\end{aligned}
\right.$\\
  \hline
  Description&
        The output image $G_{X \times Y}$ has the same size and format as the input image $F_{X \times Y}$. As a piece of 8-bit grayscale image, all of its pixels are with intensity of either 255 or 0, so it only shows the same information as a binary image.
  \\
  \hline
  Sources& \cite{Ferrari2018b} \\
  \hline
  Ref.\ By &\\
  \hline
\end{tabular}
\end{minipage}\\

~\newline

\noindent
\begin{minipage}{\textwidth}
\renewcommand*{\arraystretch}{1.5}
\begin{tabular}{| p{\colAwidth} | p{\colBwidth}|}
  \hline
  \rowcolor[gray]{0.9}
  Number& IM\refstepcounter{instnum}\theinstnum \label{IM_multifindk}\\
  \hline
  Label& \bf Otsu's Method to find the multiple optimal threshold values $k^{\star}_{1}$ and $k^{\star}_{2}$\\
  \hline
  Input& $F_{X \times Y}$ from \ddref{DD_inoutimage}, $f(x,y)$ from \ddref{DD_featurevalue}, $L$ from \ddref{DD_numberofshadesgray}, $\sigma_{B}$, k, $P_{1}$, $P_{2}$, $m_{1}$, $m_{2}$, $m_{G}$ from \ddref{DD_betweenvariance}\\
  \hline
  Output& $k^{\star}_{1}$ and $k^{\star}_{2}$, such that\\
  & $\sigma^{2}_{B}(k^{\star}_{1}, k^{\star}_{2}) = \underset{0<k_{1}<k_{2}<L-1}{max}\sigma^{2}_{B}(k_{1}, k_{2})$\\
  \hline
  Description&
        Similarly as the normal Otsu's Method,
        
        $\sigma^{2}_{B}(k_{1}, k_{2}) = P_{1}(k_{1})(m_{1}(k_{1}) - m_{G})^{2} + P_{2}(k_{2})(m_{2}(k_{2}) - m_{G})^{2} + P_{3}(k_{3})(m_{3}(k_{3}) - m_{G})^{2}$
  \\
  \hline
  Sources& \cite{Ferrari2018b} \\
  \hline
  Ref.\ By & \iref{IM_globaloutput}\\
  \hline
\end{tabular}
\end{minipage}\\

~\newline

\noindent
\begin{minipage}{\textwidth}
\renewcommand*{\arraystretch}{1.5}
\begin{tabular}{| p{\colAwidth} | p{\colBwidth}|}
  \hline
  \rowcolor[gray]{0.9}
  Number& IM\refstepcounter{instnum}\theinstnum \label{IM_multioutput}\\
  \hline
  Label& \bf Use multiple global thresholds to output segmentation image\\
  \hline
  Input& $F_{X \times Y}$ from \ddref{DD_inoutimage}, $f(x,y)$ from \ddref{DD_featurevalue}, $L$ from \ddref{DD_numberofshadesgray}, $k^{\star}_{1}$ and $k^{\star}_{2}$ from \iref{IM_multifindk}\\
  \hline
  Output& $G_{X \times Y}$, such that for pixel at each $(x,y)$,
  
  $g(x,y)=\left\{
\begin{aligned}
&255,\ &if\ f(x,y) > k^{\star}_{2} \\
&128,\ &if\ k^{\star}_{1} < f(x,y) \leq k^{\star}_{2}\\
&0,\ &if\ f(x,y) \leq k^{\star}_{1}
\end{aligned}
\right.$\\
  \hline
  Description&
        The output image $G_{X \times Y}$ has the same size and format as the input image $F_{X \times Y}$. As a piece of 8-bit grayscale image, all of its pixels are with intensity of 255, 188 or 0, so it shows more information than a binary image.
  \\
  \hline
  Sources& \cite{Ferrari2018b} \\
  \hline
  Ref.\ By &\\
  \hline
\end{tabular}
\end{minipage}\\

%~\newline
\section{Variabilities}

\subsection{Assumptions}

\begin{itemize}

\item[A\refstepcounter{assumpnum}\theassumpnum \label{A_2Dgrayscale}:]
The images are 2D and grayscale.

\item[A\refstepcounter{assumpnum}\theassumpnum \label{A_8bitinteger}:]
The pixel format of input images is DICOM image, where the feature value is the gray intensity value stored as an 12-bit or 16-bit integer giving a range of possible values from 0 to 4095 or 65535.

The pixel format of output images is the byte image, where the feature value is the gray intensity value stored as an 8-bit integer giving a range of possible values from 0 to 255. 

\end{itemize}

\subsection{Calculation} \label{sec_Calculation}

\begin{table}[h]
\begin{tabular}{|l|c|}
\hline
\textbf{Variabilities} & \multicolumn{1}{l|}{\textbf{Parameter of Variation}} \\ \hline
Allowed input & \begin{tabular}[c]{@{}c@{}}set of \{digital image, \\ binary data representing the image\}\end{tabular} \\ \hline
Dimension of spatial coordinates ($a$) & $\mathbb{N}$ \\ \hline
Dimension of feature values ($b$) & $\mathbb{N}$ \\ \hline
Number  of  the  discrete  levels  of  the  feature value ($L$) & $\mathbb{N}$ \\ \hline
Calculate methods & \begin{tabular}[c]{@{}c@{}}set of \{Global threshold method, \\ multiple threshold method\}\end{tabular} \\ \hline
\begin{tabular}[c]{@{}l@{}}Number of threshold values in Otsu’ Method\\  with multiple thresholds (number of $k_{1}, k_{2}, ...$)\end{tabular} & set of $\{1, 2, 3, ..., L-2\}$ \\ \hline
\end{tabular}
\caption{Calculation Variabilities}
\label{Tb_calvar}
\end{table}

\subsection{Output} \label{sec_Output} 
\begin{table}[h]
\begin{tabular}{|l|c|}
\hline
\textbf{Variabilities} & \multicolumn{1}{l|}{\textbf{Parameter of Variation}} \\ \hline
Output & \begin{tabular}[c]{@{}c@{}}set of \{digital image, \\ binary data representing the image\}\end{tabular} \\ \hline
Optimal threshold value ($k^{\star}$) & set of $\{1, 2, 3, ..., L-2\}$ \\ \hline
Optimal threshold values ($k^{\star}_{1}$ and $k^{\star}_{2}$) & set of $[1, k^{\star}_{2}-2]$ and $[k^{\star}_{1}+2,L-2]$ \\ \hline
\begin{tabular}[c]{@{}l@{}}Number of optimal threshold values in Otsu’ Method\\  (number of $k^{\star}_{1}, k^{\star}_{2}, ...$)\end{tabular} & set of $\{1, 2, 3, ..., L-2\}$ \\ \hline
\end{tabular}
\caption{Output Variabilities}
\label{Tb_outvar}
\end{table}

\section{Requirements}

This section provides the functional requirements, the business tasks that the
software is expected to complete, and the nonfunctional requirements, the
qualities that the software is expected to exhibit.

\subsection{Functional Requirements}

\noindent \begin{itemize}

\item[R\refstepcounter{reqnum}\thereqnum \label{R_Inputs}:] 
\famname{} shall verify that the input data are valid. A valid input image must be 2D 12-bit or 16-bit grayscale DICOM image. An error message shall be displayed if input data are invalid.

\item[R\refstepcounter{reqnum}\thereqnum \label{R_OutputInputs}:] 
\famname{} shall guarantee that the output file is the same pixel size as the input file.

\item[R\refstepcounter{reqnum}\thereqnum \label{R_Calculate}:]
\famname{} shall provide correct calculate according to Instance Models according to the user's choice of which method to use, single or multiple global thresholds. \famname{} shall also display the correct optimal threshold value(s) $k^{\star}$ or $k^{\star}_{1}$ and $k^{\star}_{2}$ accordingly. 

\item[R\refstepcounter{reqnum}\thereqnum \label{R_VerifyOutput}:]
\famname{} shall verify that the output image must be 2D 8-bit grayscale image and the pixel format must be the byte image, where the feature value must be the gray intensity value stored as an 8-bit integer giving a range of possible values from 0 to 255.

\item[R\refstepcounter{reqnum}\thereqnum \label{R_Outputk}:] 
\famname{} shall output segmentation image.

\end{itemize}

\subsection{Nonfunctional Requirements}

\begin{itemize}
\item[R\refstepcounter{reqnum}\thereqnum
\label{R_install}:]
Installability: \famname{} shall be able to be installed and uninstalled on Windows 10, macOS 10.14, and Ubuntu Linux 18.04. The installation and uninstallation process shall be easy and fast.
\item[R\refstepcounter{reqnum}\thereqnum
\label{R_correct}:]
Correctness: The output image will be generally similar to the output from VTK.
\item[R\refstepcounter{reqnum}\thereqnum
\label{R_verify}:]
Verifiability: \famname{} shall be easy to be checked or tested.
\item[R\refstepcounter{reqnum}\thereqnum
\label{R_robust}:]
Robustness: \famname{} will not crash when a user provides invalid input.
\item[R\refstepcounter{reqnum}\thereqnum
\label{R_use}:]
Usability: \famname{} shall be easy and satisfying for users to learn and use.
\item[R\refstepcounter{reqnum}\thereqnum
\label{R_maintain}:]
Maintainability: \famname{} shall be documented with an CA, VnV, MG, and MIS. It shall be able to undergo changes, like adding or changing functionality, meeting new requirements or fixing errors.
\item[R\refstepcounter{reqnum}\thereqnum
\label{R_portal}:]
Portability: \famname{} shall be able to run on Windows 10, macOS 10.14, and Ubuntu Linux 18.04.
environments.
\item[R\refstepcounter{reqnum}\thereqnum
\label{R_understand}:]
Understandability: The code shall be easy to understand, follow a coding standard and uses proper comments.
\end{itemize}

\section{Likely Changes}    

\noindent \begin{itemize}

\item[LC\refstepcounter{lcnum}\thelcnum\label{LC_localthres}:]This document only describes global threshold with Otsu's Method. It could include the method using local thresholds in the future.

\item[LC\refstepcounter{lcnum}\thelcnum\label{LC_morek}:]This document only describes multi-threshold method with 2 thresholds. It could list more Instance Models with more thresholds in the future.

\item[LC\refstepcounter{lcnum}\thelcnum\label{LC_allmethods}:]This document only specifies one segmentation method. It could list specifications for all the methods in Table \ref{fg_segmethods} in the future.

\item[LC\refstepcounter{lcnum}\thelcnum\label{LC_allanalysis}:]This document only describes segmentation. It could list specifications for all the sections in analysis in Figure \ref{fg_miafunctions} in the future.

\end{itemize}

\section{Traceability Matrices and Graphs}

The purpose of the traceability matrices is to provide easy references on what has to be additionally modified if a certain component is changed.  Every time a component is changed, the items in the column of that component that are marked with an ``X'' may have to be modified as well.  Table~\ref{Table:trace} shows the
dependencies of theoretical models, general definitions, data definitions, and instance models with each other.

\begin{table}[h!]
\centering
\begin{tabular}{|c|c|c|c|c|c|c|c|c|c|c|c|c|c|c|c|c|c|c|c|c|c|c|c|c|c|}
\hline        
	& \tref{T_mathimage}& \tref{T_globalthres}& \tref{T_multithres}& \tref{T_otsu}& \ddref{DD_digitalimage} & \ddref{DD_2Dimage}& \ddref{DD_inoutimage} & \ddref{DD_featurevalue}& \ddref{DD_numberofshadesgray}& \ddref{DD_thresvalue}& \ddref{DD_betweenvariance}& \iref{IM_otsufindk}& \iref{IM_globaloutput}& \iref{IM_multifindk}& \iref{IM_multioutput} \\
\hline
\tref{T_mathimage}     & & & & & & & & & & & & & & & \\ \hline
\tref{T_globalthres}     & & & & & & &X &X &X &X & & & & & \\ \hline
\tref{T_multithres}     & &X & & & & &X &X &X &X & & & & & \\ \hline
\tref{T_otsu}        & & & & & & & & &X & &X & & & & \\ \hline
\ddref{DD_digitalimage}      &X & & & & & & & & & & & & & & \\ \hline
\ddref{DD_2Dimage} & & & & &X & & & & & & & & & & \\ \hline
\ddref{DD_inoutimage}  & & & & & &X & &X & & & &X &X &X &X \\ \hline
\ddref{DD_featurevalue}    & & & & & & & & &X & & &X &X &X &X \\ \hline
\ddref{DD_numberofshadesgray}     & & & & & & & & & & & &X &X &X &X \\ \hline
\ddref{DD_thresvalue}    & & & & & & & & &X & & & & & & \\ \hline
\ddref{DD_betweenvariance}     & &X & & & & & &X &X &X & &X & &X & \\ \hline
\iref{IM_otsufindk}      & & & &X & & &X &X &X &X &X & &X & & \\ \hline
\iref{IM_globaloutput}      & &X & & & & & & & & & &X & &X & \\ \hline
\iref{IM_multifindk}    & & &X &X & & & & & & &X & & & &X \\ \hline
\iref{IM_multioutput}    & & & & & & & & & & & & & & & \\
\hline
\end{tabular}
\caption{Traceability Matrix Showing the Connections Between Items of Different Sections}
\label{Table:trace}
\end{table}

\newpage

\bibliographystyle {plainnat}
\bibliography {../../refs/References}

\end{document}
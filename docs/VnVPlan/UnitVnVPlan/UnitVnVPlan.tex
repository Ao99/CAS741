\documentclass[12pt, titlepage]{article}

\input{../../Comments}
%% Common Parts
\usepackage{booktabs}
\usepackage{tabularx}
\usepackage{hyperref}
\usepackage[round]{natbib}

\newcommand{\famname}{MIA}
\newcommand{\progname}{MISEG}

\hypersetup{
    bookmarks=true,         % show bookmarks bar?
    colorlinks=true,       % false: boxed links; true: colored links
    linkcolor=red,          % color of internal links (change box color with linkbordercolor)
    citecolor=green,        % color of links to bibliography
    filecolor=magenta,      % color of file links
    urlcolor=cyan           % color of external links
}


\begin{document}

\title{Project Title: Unit Verification and Validation Plan for \progname{}} 
\author{Ao Dong}
\date{\today}
	
\maketitle

\pagenumbering{roman}

\section{Revision History}

\begin{tabularx}{\textwidth}{p{3cm}p{2cm}X}
\toprule {\bf Date} & {\bf Version} & {\bf Notes}\\
\midrule
Dec 14 & 1.0 & Initial Draft\\
\bottomrule
\end{tabularx}

~\newpage

\tableofcontents

\listoftables

\newpage

\section{Symbols, Abbreviations and Acronyms}

\renewcommand{\arraystretch}{1.2}
\begin{tabular}{l l} 
  \toprule		
  \textbf{symbol} & \textbf{description}\\
  \midrule 
  T & Test\\
  \bottomrule
\end{tabular}\\

For the other symbols, abbreviations and acronym, see SRS Documentation at
\url{https://github.com/Ao99/MIA/blob/master/docs/SRS/SRS.pdf}

\newpage

\pagenumbering{arabic}

This document describes procedures concerning the unit testing for \progname{}
for compliance with the requirements.

Some general information such as introduction to the software and testing
scopes are included in Section \ref{sec_geinfo}. Verification plans and
test tools are in Section \ref{sec_plan}.

\section{General Information}
\label{sec_geinfo}

\subsection{Purpose}

The software going through the test is Medical Imaging Segmentation
(\progname{}).

Segmentation, separation of structures of interest from the background and from
each other~\cite{Bankman2000}. Image segmentation is the process of
partitioning an image into different meaningful segments. In medical imaging,
these segments
often correspond to different tissue classes, organs, pathologies, or other
biologically relevant structures~\cite{Forouzanfar2010}.

\progname{} uses one of many segmentation algorithms - the Intensity Threshold
method. It also uses Otsu's Method to find the optimal threshold value(s).
After receiving input medical image from the users, \progname{} calculates the
optimal threshold value(s), and output the processed segmentation image.

\subsection{Scope}

The Input Module relies on a library developed by others, so it will not be
tested within the scope of this document.

The modules that are going to be tested are Image Data Structure, Optimal
Thresholds Calculation, Output Module and Image Verification.

\section{Plan}
\label{sec_plan}
	
\subsection{Verification and Validation Team}

\begin{itemize}
\item Ao Dong
\item Prof.\ Spencer Smith
\item Peter Michalski
\item Zhi Zhang
\item Sasha Soraine
\item Sharon Wu
\end{itemize}

\subsection{Automated Testing and Verification Tools}

The Java unit testing framework JUnit will be used to test the modules.

\subsection{Non-Testing Based Verification}

The non-testing based verification will be done in the System VnV.

\section{Unit Test Description}

All modules should meet the specifications defined in MIS \cite{Dong2019MIS}.

\subsection{Tests for Functional Requirements}

\subsubsection{Image Data Structure}
\label{sec_imageData}

This is an abstract data type module. It should correctly construct a data type
instance if valid parameters are given, and throw exceptions if invalid ones are
given.

\begin{enumerate}

\item{Valid parameter\\}

Type: Automatic
					
Initial State: none.
					
Input: SYMBOLIC\_CONSTANTS.validSize, SYMBOLIC\_CONSTANTS.validSize,
SYMBOLIC\_CONSTANTS.validSequence
					
Output: A message saying that a new ImageData is successfully constructed, with
the width, length and pixel value sequence displayed.

Test Case Derivation: A new ImageData is successfully constructed and no
exception is thrown.

How test will be performed: It will be performed by test classes built with the
help of JUnit.
					
\item{Invalid parameter1\\}

Type: Automatic
					
Initial State: none.
					
Input: SYMBOLIC\_CONSTANTS.negSize, SYMBOLIC\_CONSTANTS.validSize,
SYMBOLIC\_CONSTANTS.validSequence
					
Output: A message saying that a new ImageData is not successfully constructed,
with the badWidthInput error message from MIS shown.

Test Case Derivation: A new ImageData is not successfully constructed and the
correct exception is thrown.

How test will be performed: It will be performed by test classes built with the
help of JUnit.

\item{Invalid parameter2\\}

Type: Automatic
					
Initial State: none.
					
Input: SYMBOLIC\_CONSTANTS.validSize, SYMBOLIC\_CONSTANTS.negSize,
SYMBOLIC\_CONSTANTS.validSequence
					
Output: A message saying that a new ImageData is not successfully constructed,
with the badWidthInput error message from MIS shown.

Test Case Derivation: A new ImageData is not successfully constructed and the
correct exception is thrown.

How test will be performed: It will be performed by test classes built with the
help of JUnit.

\item{Invalid parameter3\\}

Type: Automatic
					
Initial State: none.
					
Input: SYMBOLIC\_CONSTANTS.validSize2, SYMBOLIC\_CONSTANTS.validSize2,
SYMBOLIC\_CONSTANTS.validSequence
					
Output: A message saying that a new ImageData is not successfully constructed,
with the badPixelValueLength error message from MIS shown.

Test Case Derivation: A new ImageData is not successfully constructed and the
correct exception is thrown.

How test will be performed: It will be performed by test classes built with the
help of JUnit.

\end{enumerate}

\subsubsection{Optimal Thresholds Calculation}
\label{sec_ThresCal}

This module should update chosenThresNum according to user's choice, and show an
error message if invalid input is given.

\begin{enumerate}

\item{Valid choice input\\}

Type: Automatic
					
Initial State: none.
					
Input: SYMBOLIC\_CONSTANTS.validChoice
					
Output: A message saying that chosenThresNum is successfully updated, with the
value of chosenThresNum displayed.

Test Case Derivation: chosenThresNum is successfully updated and no exception is
thrown.

How test will be performed: It will be performed by test classes built with the
help of JUnit.
					
\item{Inalid choice input1\\}

Type: Automatic
					
Initial State: none.
					
Input: SYMBOLIC\_CONSTANTS.invalidChoice1
					
Output: A message saying that chosenThresNum is not successfully updated, with
the badChoiceInput error message from MIS shown.

Test Case Derivation: chosenThresNum is not successfully updated and the correct
exception is thrown.

How test will be performed: It will be performed by test classes built with the
help of JUnit.

\item{Inalid choice input2\\}

Type: Automatic
					
Initial State: none.
					
Input: SYMBOLIC\_CONSTANTS.invalidChoice2
					
Output: A message saying that chosenThresNum is not successfully updated, with
the badChoiceInput error message from MIS shown.

Test Case Derivation: chosenThresNum is not successfully updated and the correct
exception is thrown.

How test will be performed: It will be performed by test classes built with the
help of JUnit.

\end{enumerate}

\subsubsection{Output Module}
\label{sec_output}

This module should correctly segment input images.

\begin{enumerate}

\item{Segmentation\\}

Type: Automatic
					
Initial State: none.
					
Input: SYMBOLIC\_CONSTANTS.validImg1
					
Output: A correctly segmented instance of ImageData, with width 100, height 100,
and a valid pixelValue which is a sequence of $\mathbf{N}$ with elements in
$[0,255]$
This sequence should be the same as the one documented in This sequence is
documented in test/segImage.txt

Test Case Derivation: segImage is successfully updated and the result is
correct.

How test will be performed: It will be performed by test classes built with the
help of JUnit.

\end{enumerate}

\subsubsection{Image Verification}
\label{sec_imageVerify}

This module should correctly return verification result and throw proper
exceptions.

\begin{enumerate}

\item{Valid image\\}

Type: Automatic
					
Initial State: none.
					
Input: SYMBOLIC\_CONSTANTS.validImg1
					
Output: Boolean true and no error message.

Test Case Derivation: a valid instance of ImageData is correctly identified.

How test will be performed: It will be performed by test classes built with the
help of JUnit.

\item{Invalid image1\\}

Type: Automatic
					
Initial State: none.
					
Input: SYMBOLIC\_CONSTANTS.invalidImg1
					
Output: Boolean false, with the badSize error message from MIS shown.

Test Case Derivation: an invalid instance of ImageData is correctly identified
and the proper error message is shown.

How test will be performed: It will be performed by test classes built with the
help of JUnit.

\item{Invalid image2\\}

Type: Automatic
					
Initial State: none.
					
Input: SYMBOLIC\_CONSTANTS.invalidImg2
					
Output: Boolean false, with the badPixelData error message from MIS shown.

Test Case Derivation: an invalid instance of ImageData is correctly identified
and the proper error message is shown.

How test will be performed: It will be performed by test classes built with the
help of JUnit.

\item{Consistent sizes\\}

Type: Automatic
					
Initial State: none.
					
Input: SYMBOLIC\_CONSTANTS.validImg1, SYMBOLIC\_CONSTANTS.validImg1
					
Output: Boolean true and no error message.

Test Case Derivation: two instances of ImageData with the same size are
correctly identified.

How test will be performed: It will be performed by test classes built with the
help of JUnit.

\item{Inconsistent sizes\\}

Type: Automatic
					
Initial State: none.
					
Input: SYMBOLIC\_CONSTANTS.validImg1, SYMBOLIC\_CONSTANTS.validImg2
					
Output: Boolean false, with the badSize2 error message from MIS shown.

Test Case Derivation: two instances of ImageData with different sizes are
correctly identified and the proper error message is shown.

How test will be performed: It will be performed by test classes built with the
help of JUnit.

\end{enumerate}

\subsection{Tests for Nonfunctional Requirements}

The performance of module Optimal Thresholds Calculation will be tested.

\subsubsection{Optimal Thresholds Calculation}
\label{sec_thresCalPerformance}

\begin{enumerate}

\item{Max thresholds calculation speed\\}

Type: Automatic
					
Initial State: none
					
Input: SYMBOLIC\_CONSTANTS.maxThresNum
					
Output: Time used to finish the optimal thresholds calculation.
					
How test will be performed: It will be performed by test classes built with the
help of JUnit.

\end{enumerate}

\subsection{Traceability Between Test Cases and Modules}

The purpose of the traceability matrices is to provide easy references on what
has to be additionally modified if a certain component is changed. Table
\ref{Tb_trace} shows the
dependencies between the test cases and the requirements.

\begin{table}[h]
\centering
\begin{tabular}{@{}ll@{}}
\toprule
Test Cases & Modules \\ \midrule
\ref{sec_imageData} & Image Data Structure \\
\ref{sec_ThresCal}, \ref{sec_thresCalPerformance} & Optimal Thresholds
Calculation \\
\ref{sec_output} & Output Module \\
\ref{sec_imageVerify} & Image Verification \\ \bottomrule
\end{tabular}
\caption{Traceability Matrix showing the connections between modules and
tests}
\label{Tb_trace}
\end{table}

\newpage

\bibliographystyle{plainnat}
\bibliography{../../../refs/References}

\newpage

\section{Appendix}

\subsection{Symbolic Parameters}

\begin{itemize}
\item SYMBOLIC\_CONSTANTS.negSize = -1
\item SYMBOLIC\_CONSTANTS.validSize = 100
\item SYMBOLIC\_CONSTANTS.validSize2 = 200
\item SYMBOLIC\_CONSTANTS.validSequence = $<1,1,...,1>$ which is a sequence of
$\mathbf{N}$ with length of 10000, and every element is 1.
\item SYMBOLIC\_CONSTANTS.validChoice = 2
\item SYMBOLIC\_CONSTANTS.invalidChoice1 = 6
\item SYMBOLIC\_CONSTANTS.invalidChoice2 = A
\item SYMBOLIC\_CONSTANTS.validImg1, an instance of ImageData, with width 100,
height 100, and a valid pixelValue which is a sequence of $\mathbf{N}$ with
elements in $[0,255]$. This sequence is documented in test/inputImage.txt
\item SYMBOLIC\_CONSTANTS.validImg2, an instance of ImageData, with width 200,
height 200, and a valid pixelValue which is a sequence of $\mathbf{N}$ with
elements in $[0,255]$.
\item SYMBOLIC\_CONSTANTS.invalidImg1, an instance of ImageData, with width -1,
height -1, and a valid pixelValue which is a sequence of $\mathbf{N}$ with
elements in $[0,255]$.
\item SYMBOLIC\_CONSTANTS.invalidImg2, an instance of ImageData, with width 100,
height 100, and a valid pixelValue which is a sequence of $\mathbf{N}$
containing elements -1 and 256.
\item SYMBOLIC\_CONSTANTS.maxThresNum = 2.
\end{itemize}

\end{document}
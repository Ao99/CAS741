\documentclass[12pt, titlepage]{article}

%% Comments

\usepackage{color}

\newif\ifcomments\commentstrue

\ifcomments
\newcommand{\authornote}[3]{\textcolor{#1}{[#3 ---#2]}}
\newcommand{\todo}[1]{\textcolor{red}{[TODO: #1]}}
\else
\newcommand{\authornote}[3]{}
\newcommand{\todo}[1]{}
\fi

\newcommand{\wss}[1]{\authornote{blue}{SS}{#1}} 
\newcommand{\plt}[1]{\authornote{magenta}{TPLT}{#1}} %For explanation of the template
\newcommand{\an}[1]{\authornote{cyan}{Author}{#1}}

%% Common Parts
\usepackage{booktabs}
\usepackage{tabularx}
\usepackage{hyperref}
\usepackage[round]{natbib}
\usepackage{amsmath, mathtools}
\usepackage{amsfonts}
\usepackage{amssymb}
\usepackage{graphicx}
\usepackage{colortbl}
\usepackage{xr}
\usepackage{longtable}
\usepackage{xfrac}
\usepackage{float}
\usepackage{siunitx}
\usepackage{caption}
\usepackage{pdflscape}
\usepackage{afterpage}
\usepackage{tikz}
\usetikzlibrary{mindmap}
\usepackage{multirow}
\usepackage{fullpage}
%\usepackage{refcheck}

\newcommand{\famname}{MIA}
\newcommand{\progname}{MISEG}

% For easy change of table widths
\newcommand{\colZwidth}{1.0\textwidth}
\newcommand{\colAwidth}{0.13\textwidth}
\newcommand{\colBwidth}{0.82\textwidth}
\newcommand{\colCwidth}{0.1\textwidth}
\newcommand{\colDwidth}{0.05\textwidth}
\newcommand{\colEwidth}{0.8\textwidth}
\newcommand{\colFwidth}{0.17\textwidth}
\newcommand{\colGwidth}{0.5\textwidth}
\newcommand{\colHwidth}{0.28\textwidth}

% Used so that cross-references have a meaningful prefix
\newcounter{defnum} %Definition Number
\newcommand{\dthedefnum}{GD\thedefnum}
\newcommand{\dref}[1]{GD\ref{#1}}
\newcounter{datadefnum} %Datadefinition Number
\newcommand{\ddthedatadefnum}{DD\thedatadefnum}
\newcommand{\ddref}[1]{DD\ref{#1}}
\newcounter{theorynum} %Theory Number
\newcommand{\tthetheorynum}{T\thetheorynum}
\newcommand{\tref}[1]{T\ref{#1}}
\newcounter{tablenum} %Table Number
\newcommand{\tbthetablenum}{T\thetablenum}
\newcommand{\tbref}[1]{TB\ref{#1}}
\newcounter{assumpnum} %Assumption Number
\newcommand{\atheassumpnum}{P\theassumpnum}
\newcommand{\aref}[1]{A\ref{#1}}
\newcounter{goalnum} %Goal Number
\newcommand{\gthegoalnum}{P\thegoalnum}
\newcommand{\gsref}[1]{GS\ref{#1}}
\newcounter{instnum} %Instance Number
\newcommand{\itheinstnum}{IM\theinstnum}
\newcommand{\iref}[1]{IM\ref{#1}}
\newcounter{reqnum} %Requirement Number
\newcommand{\rthereqnum}{P\thereqnum}
\newcommand{\rref}[1]{R\ref{#1}}
\newcounter{lcnum} %Likely change number
\newcommand{\lthelcnum}{LC\thelcnum}
\newcommand{\lcref}[1]{LC\ref{#1}}

\hypersetup{
    bookmarks=true,         % show bookmarks bar?
    colorlinks=true,       % false: boxed links; true: colored links
    linkcolor=red,          % color of internal links (change box color with linkbordercolor)
    citecolor=green,        % color of links to bibliography
    filecolor=magenta,      % color of file links
    urlcolor=cyan           % color of external links
}


\externaldocument{../../SRS/CA}

\begin{document}

\title{Module Interface Specification for \progname}

\author{Ao Dong}

\date{\today}

\maketitle

\pagenumbering{roman}

\section{Revision History}

\begin{tabularx}{\textwidth}{p{3cm}p{2cm}X}
\toprule {\bf Date} & {\bf Version} & {\bf Notes}\\
\midrule
Nov 25 & 1.0 & Initial Draft\\
Dec 11 & 1.1 & Modification according to feedback\\
Dec 12 & 1.2 & Modification according to implementation\\
\bottomrule
\end{tabularx}

~\newpage

\section{Symbols, Abbreviations and Acronyms}

See SRS Documentation at
\url{https://github.com/Ao99/MIA/blob/master/docs/SRS/SRS.pdf}

\newpage

\tableofcontents

\newpage

\pagenumbering{arabic}

\section{Introduction}

The following document details the Module Interface Specifications for
\progname{} which is for medical image segmentation.

Complementary documents include the System Requirement Specifications
and Module Guide.  The full documentation and implementation can be
found at \url{https://github.com/Ao99/MIA}.

\section{Notation}

The structure of the MIS for modules comes from \citet{HoffmanAndStrooper1995},
with the addition that template modules have been adapted from
\cite{GhezziEtAl2003}.  The mathematical notation comes from Chapter 3 of
\citet{HoffmanAndStrooper1995}.  For instance, the symbol := is used for a
multiple assignment statement and conditional rules follow the form $(c_1
\Rightarrow r_1 | c_2 \Rightarrow r_2 | ... | c_n \Rightarrow r_n )$. This
document has one modification to the original notations: the concatenation
notation $||$ can be used to build a new sequence from an existing sequence.
Forexample, s2 $:= ||(x : \mathbb{N} | x \in s1 \cdot x+1)$, where s1 =
$\langle1,2,...,10 \rangle$, then s2 = $\langle 2,3,...,11 \rangle$. We adopt
the
notation in this way assuming that the elements in a sequence are in order
rather than random. Using it in this way is for convenience and practical
reasons, but not for properly expanding the original notation in a mathematical
way. \wss{explain that
  the notation is for convenience.  It is not proper mathematics; the order
  would of the elements is actually random.  The notation is adopted for
  practical reasons.}
\an{Explanation added.}

The following table summarizes the primitive data types used by \progname. 

\begin{center}
\renewcommand{\arraystretch}{1.2}
\noindent 
\begin{tabular}{l l p{7.5cm}} 
\toprule 
\textbf{Data Type} & \textbf{Notation} & \textbf{Description}\\ 
\midrule
character & char & a single symbol or digit\\
integer & $\mathbb{Z}$ & a number without a fractional component in (-$\infty$,
$\infty$) \\
natural number & $\mathbb{N}$ & a number without a fractional component in [1,
$\infty$) \\
real & $\mathbb{R}$ & any number in (-$\infty$, $\infty$)\\
boolean & boolean & a value in $\{true, false\}$\\
\bottomrule
\end{tabular} 
\end{center}

The following table summarizes other data types used by \progname.
\begin{center}
\renewcommand{\arraystretch}{1.2}
\noindent 
\begin{tabular}{l l p{7.5cm}} 
\toprule 
\textbf{Data Type} & \textbf{Notation} & \textbf{Description}\\ 
\midrule
DICOM file & inputFile & a DICOM image file\\
DICOM frame & dcmFrame & a frame of image in a DICOM image file\\
image data & imageData & a data structure containing width, height and a
sequence of pixel values\\
bitmap file & outputFile & an 8-bit 2D grayscale bitmap image file\\
\bottomrule
\end{tabular} 
\end{center}

\noindent
The specification of \progname \ uses some derived data types: sequences,
strings, and
tuples. Sequences are lists filled with elements of the same data type. Strings
are sequences of characters. Tuples contain a list of values, potentially of
different types. In addition, \progname \ uses functions, which
are defined by the data types of their inputs and outputs. Local functions are
described by giving their type signature followed by their specification.

\section{Module Decomposition}

The following table is taken directly from the Module Guide document for this
project.

\begin{table}[h!]
\centering
\begin{tabular}{p{0.3\textwidth} p{0.6\textwidth}}
\toprule
\textbf{Level 1} & \textbf{Level 2}\\
\midrule

{Hardware-Hiding Module} & ~ \\
\midrule

\multirow{6}{0.3\textwidth}{Behaviour-Hiding Module}
& Input Module\\
& Output Module\\
& Optimal Thresholds Calculation\\
& Image Verification\\
& Constant Values\\
& Control Module\\
\midrule

\multirow{2}{0.3\textwidth}{Software Decision Module}
& Sequence Data Structure\\
& Image Data Structure\\
\bottomrule

\end{tabular}
\caption{Module Hierarchy}
\label{TblMH}
\end{table}

\newpage

\section{MIS of Control Module} \label{Md_Control}
\subsection{Module}
main
\subsection{Uses}

Input in Section \ref{Md_Input}, ThresCal in Section \ref{Md_Calculation},
Output in Section \ref{Md_Output}

\subsection{Syntax}

\subsubsection{Exported Constants}

\subsubsection{Exported Access Programs}

\begin{center}
\begin{tabular}{p{2cm} p{2cm} p{2cm} p{6cm}}
\hline
\textbf{Name} & \textbf{In} & \textbf{Out} & \textbf{Exceptions} \\
\hline
main & - & - & \\
\hline
\end{tabular}
\end{center}

\subsection{Semantics}

\subsubsection{State Variables}

None

\subsubsection{Environment Variables}

None

\subsubsection{Assumptions}

None

\subsubsection{Access Routine Semantics}

\noindent main():
\begin{itemize}
\item transition: use other modules by following these steps
\begin{enumerate}
\item Get ($filenameIn$: string) and ($filenameOut$: string) from user
\item Input.loadInput($filenameIn$) \wss{You filename is sometimes in italics
and sometimes not. You should be consistent.} \an{I put it this way because it
is a parameter. I changed all parameters to italics everywhere}
\item For ($j: \mathbb{N}$) from 0 to Input.numFrames, repeat the following
steps,
\item ThresCal.calculation($j$)
\item Output.displayThresholds()
\wss{You do not actually have to compare with true.
  If ThresCal.isThresValid is a Boolean, it is already True or False.  This
comment applies elsewhere in this spec as well.}\an{I made some change so this
part doesn't compare True or False. But for the rest of the document, I
modifiedas you suggested.}
\item Output.writeOutput($filenameOut$)
\wss{Our notation does not have an ``else''.  You can just say True in place of
``else''. This same comment applies throughout this document.}\an{I made some
change so this part doesn't compare True or False. But for the rest of the
document, I modified as you suggested.}
\end{enumerate}
\item output: none
\item exception: none
\end{itemize}

\wss{Your exception covers the same ground as given in your pseudo code.  You
shouldn't really repeat this here, especially since it is ambiguous in the
wayyou have worded it. I suggest you replace the exception information you have
with a note that points the reader to the transition field for the definition
ofthe exception behaviour.}
\an{I've moved all exception to other modules, so not anymore a problem here,
but I'll also make sure not a problem in other parts.}

\subsubsection{Local Functions}

None

\newpage

\section{MIS of Constant Values} \label{Md_Constants}
\subsection{Module}
Constants
\subsection{Uses}

None

\subsection{Syntax}

\subsubsection{Exported Constants}

numsThres: set of $\mathbb{N}$\\
numsThres := \{1, 2, 3\}

\wss{Are these constants all necessary?  Are the bounds really more related to
  available memory?}
\an{I removed bounds for images. Now images are valid with width and height
greater than 0}
\wss{For the choices, I suggest that you introduce an enumerated type (via an
exported type, like the department names in the student allocation example). The
names choice1 and choice2 do not really tell the reader anything. Can you
  give them more meaningful names?}
\an{I only kept a set of natural number here. When the user's choice is in this
set, it's valid}

\subsubsection{Access Routine Semantics}
N/A

\newpage

\section{MIS of Input Module} \label{Md_Input}
\subsection{Module}

Input

\subsection{Uses}

ImageData \ref{Md_ImageDS}, ImageVerify in Section \ref{Md_Verify}

\subsection{Syntax}

\subsubsection{Exported Constants}

\subsubsection{Exported Access Programs}

\begin{center}
\begin{tabular}{p{3cm} p{2cm} p{5cm} p{3cm}}
\hline
\textbf{Name} & \textbf{In} & \textbf{Out} & \textbf{Exceptions} \\
\hline
loadInput & $s$: string & - & badInputFileName, badInputFileFormat\\
verifyInput & & - & emptyFrameJ \\
loadedImages & - & sequence of imageData\\
numFrames & - & $\mathbb{N}$ & \\
isLoaded & - & sequence of boolean\\
\hline
\end{tabular}
\end{center}

\subsection{Semantics}

\subsubsection{State Variables}

loadedImages: sequence of imageData\\
numFrames: $\mathbb{N}$\\
isLoaded: sequence of boolean

\subsubsection{Environment Variables}

inputFile: a .dcm or .dcm30 DICOM medical image file

\subsubsection{Assumptions}

The data type String has a method parseToNum() to parse a string (such as ``1")
to an $\mathbb{N}$ (such as 1).

\subsubsection{Access Routine Semantics}

\noindent Input.loadedImages:
\begin{itemize}
\item output: $out:=$ loadedImages 
\item exception: none 
\end{itemize}

\noindent Input.numFrames:
\begin{itemize}
\item output: $out:=$ numFrames 
\item exception: none 
\end{itemize}

\noindent Input.isLoaded:
\begin{itemize}
\item output: $out:=$ isLoaded 
\item exception: none 
\end{itemize}

\noindent loadInput($s$):
\begin{itemize}
\item transition: 
The filename $s$ is first associated with the file f. inputFile is used to
modify
the state variables using the following procedural specification:
\begin{enumerate}
    \item Read the inputFile.
    \item
    numFrames := information from inputFile
    \item
loadedImages := $||$($f$ : dcmFrame $|$ $f \in $ inputFile $\cdot$
dcmToImage($f$))
    \item
    verifyInput()
\end{enumerate}
\item output: none
\item exception: $exc :=$\\
a file name $s$ cannot be found
$\implies$ badInputFileName\\
the format of
inputFile is incorrect $\implies$ badInputFileFormat\\
\end{itemize}

\noindent verifyInput():
\begin{itemize}
\item transition: 
This function modifies the state variables using the following procedural
specification:
\begin{enumerate}
\item
isLoaded := $||$($img$ : imageData $|$ $img \in$ loadedImages $\cdot$
ImageVerify.verifyImageData($img$))
\item Initiate a local variable cnt: $\mathbb{N}$
\item cnt := $+$($b$ : boolean $|$ $b \in$ isLoaded $\cdot$
($b \implies$ 1 $|$ True $\implies$ 0)
\item print ``cnt image frames have been loaded".
\end{enumerate}
\item output: none 
\item exception: $exc :=$\\
$\neg$ ImageVerify.verifyImageData(loadedImages[$j$]) $\implies$ emptyFrameJ
\end{itemize}

\wss{You have a parallel data structure for loadedImages and isLoaded. They are
two separate sequences. They both need to be indexed separately. What about
  having the index in this module.  You can just give an index and it will
  return an image, as long as the image is loaded.  There could be an exception
  otherwise.  Something to think about.}
\an{I don't quite understand this comment here. It seems like what I've already
being doing. For now}

\subsubsection{Local Functions}

\noindent dcmToImage: dcmFrame $\rightarrow$ imageData\\
\noindent dcmToImage($f$) $\equiv$ new ImageData($f.x$, $f.y$,
stringToSequence($f.s$))

\noindent stringToSequence: string $\rightarrow$ sequence of $\mathbb{N}$

\noindent stringToSequence($str$) $\equiv$
$||$ ($pv$: string $|$ $pv \in$ dcmFrame $\cdot$ String.parseToNum($pv$))\\
$pv$ is a string containing grayscale value for one pixel.

\newpage

\section{MIS of Image Data Structure} \label{Md_ImageDS}
\subsection{Generic Template Module}

ImageData

\subsection{Uses}

none

\subsection{Syntax}

\subsubsection{Exported Constants}

\subsubsection{Exported Types}

ImageData = ?

\subsubsection{Exported Access Programs}

\begin{center}
\begin{tabular}{p{3cm} p{4cm} p{3cm} p{5cm}}
\hline
\textbf{Name} & \textbf{In} & \textbf{Out} & \textbf{Exceptions} \\
\hline
new ImageData & $x: \mathbb{N}$, $y: \mathbb{N}$, $p$: sequence of $\mathbb{N}$
&
imageData & badWidthInput, badHeightInput, badPixelValueLength\\
width & - & $\mathbb{N}$\\
height & - & $\mathbb{N}$ & \\
pixelValue & - & sequence of $\mathbb{N}$\\
\hline
\end{tabular}
\end{center}

\subsection{Semantics}

\subsubsection{State Variables}

width: $\mathbb{N}$\\
height: $\mathbb{N}$\\
pixelValue: sequence

\subsubsection{Environment Variables}

none

\subsubsection{Assumptions}

imageData with 0 width or height is allowed. The software uses such imageData
torepresent an empty image without successfully loading information from a
frame.
Image Verification Module \ref{Md_Verify} is able to identify it as not
successfully loaded.

\subsubsection{Access Routine Semantics}

\noindent Input.width:
\begin{itemize}
\item output: $out:=$ width 
\item exception: none 
\end{itemize}

\noindent Input.height:
\begin{itemize}
\item output: $out:=$ height 
\item exception: none 
\end{itemize}

\noindent Input.pixelValue: \wss{This access program is not given in the
syntax.The state variable is missing. Is this a copy-paste error?}
\an{It's a copy-paste error. I changed it to the correct one}
\begin{itemize}
\item output: $out:=$ pixelValue 
\item exception: none 
\end{itemize}

\noindent new ImageData($x, y, p$):
\begin{itemize}
\item transition: 
The parameters $x$ and $y$ are natural numbers, $p$ is a sequence of natural
numbers representing the pixel values from left to right, top to bottom.
ImageData() is the conxtructor of this data structure, and it modifies
the state variables using the following procedural specification:
\begin{enumerate}
    \item
    height $:= x$
    \item
    height $:= y$
    \item
    pixelValue $:= p$
\end{enumerate}
\item output: := self
\item exception: $exc :=$\\
$x < 0 \implies$ badWidthInput\\
$y < 0 \implies$ badHeightInput\\
$p.$length $\not= x \times y \implies$ badPixelValueLength
\end{itemize}

\subsubsection{Local Functions}
None

\newpage

\section{MIS of Optimal Thresholds Calculation} \label{Md_Calculation}

\subsection{Module}

ThresCal

\subsection{Uses}

Constants \ref{Md_Constants}, Input in Section \ref{Md_Input}, Image Data
Structure \ref{Md_ImageDS}

\subsection{Syntax}

\subsubsection{Exported Constants}

\subsubsection{Exported Access Programs}

\begin{center}
\begin{tabular}{p{4cm} p{2cm} p{2cm} p{6cm}}
\hline
\textbf{Name} & \textbf{In} & \textbf{Out} & \textbf{Exceptions} \\
\hline
calculation & $j: \mathbb{N}$ & - & skipCalculation, badResult\\
getUserChoice & $c: \mathbb{N}$ & - & badChoiceInput \\
frameIndex & - & $\mathbb{N}$ &  \\
chosenThresNum & - & $\mathbb{N}$ &  \\
isThresValid & - & boolean &\\
k1 & - & $\mathbb{N}$ &  \\
k2 & - & $\mathbb{N}$ &  \\
\hline
\end{tabular}
\end{center}

\subsection{Semantics}

\subsubsection{State Variables}

frameIndex: $\mathbb{N}$\\
chosenThresNum: string\\
isThresValid: boolean\\
k1: $\mathbb{N}$\\
k2: $\mathbb{N}$\\
img: imageData 

\subsubsection{Environment Variables}

None

\subsubsection{Assumptions}

None

\subsubsection{Access Routine Semantics}

\noindent ThresCal.frameIndex:
\begin{itemize}
\item output: $out :=$ frameIndex
\item exception: none 
\end{itemize}

\noindent ThresCal.chosenThresNum:
\begin{itemize}
\item output: $out :=$ chosenThresNum
\item exception: none 
\end{itemize}

\noindent ThresCal.isThresValid:
\begin{itemize}
\item output: $out :=$ isThresValid
\item exception: none 
\end{itemize}

\noindent ThresCal.k1:
\begin{itemize}
\item output: $out :=$ k1
\item exception: none 
\end{itemize}

\noindent ThresCal.k2:
\begin{itemize}
\item output: $out :=$ k2
\item exception: none 
\end{itemize}

\noindent calculation($j$):\\
$j \in \mathbb{N}$ is the index of one imageData in the Input.loadedImages
sequence.
\begin{itemize}
\item transition: \wss{I don't think you need pseudo-code for this.  A state
    based specification should be possible.}
\an{I deleted some pseudo-code and put more statements here}
This function modifies
the state variables using the following procedural specification:
\begin{enumerate}
\item frameIndex $:= j$
\item If $\neg$ Input.isLoaded[$j$], set isThresValid to False, and terminate
this method.
\item img := Input.loadedImages[$j$]
\item getUserChoice()
\item According to the chosen threshold numbers, calculate k1 (and k2):\\
if chosenThresNum = 1, use the following formula to calculate k1,\\
sigma2b1(k1)= $\max\limits_{0< t1<255}$ sigma2b1($t1$);\\
if chosenThresNum = 2, use the following formula to calculate k1 and k2,\\
sigma2b2(k1,k2) =
$\max\limits_{0<t1<t2<255}$ sigma2b2($t1,t2$)).
\item According to the chosen threshold numbers, set isThresValid to True if
thecalculated results follow the rules accordingly:\\
$1 \le k1 \le 254$\\
$1 \le k1 < k2 \le 254$
\end{enumerate}
\item output: none
\item exception: $exc :=$\\
$\neg$ Input.isLoaded[$j$] $\implies$ skipCalculation\\
chosenThresNum = 1 $\land \neg (1 \le k1 \le 254) \implies$
badResult\\
chosenThresNum = 2 $\land \neg (1 \le k1 < k2 \le 254) \implies$ badResult
\end{itemize}

\noindent getUserChoice($c$):
\begin{itemize}
\item transition:
The parameter $c$ is a natural number representing user's choice. This function
modifies the state variables using the following procedural specification:
\begin{enumerate}
\item While chosenThresNum $\not\in$ Constants.numsThres, repeat the following
two steps, until get proper choice input from the user.
\item Use hardware to display a message, asking for user's choice from
Constants.numsThres
for number of thresholds.
\item If $c \in$ Constants.numsThres, chosenThresNum := $c$
\end{enumerate}
\item output: none
\item exception: $exc := c \not\in$ Constants.numsThres
$\implies$ badChoiceInput
\end{itemize}

\subsubsection{Local Functions}

n: $\mathbb{N} \rightarrow \mathbb{N}$\\
n($i) \equiv +(pv.\mathbb{N} |pv \in$ img.pixelValue $\cdot
(pv = i \implies 1 |$ True $\implies 0))$\\
p: $\mathbb{N} \rightarrow \mathbb{R}$\\
p($i) \equiv n(i)/$(img.width $\times$ img.height)\\
prb: $\mathbb{N} \times \mathbb{N} \rightarrow \mathbb{R}$\\
prb($start,end) \equiv +(i.\mathbb{N} | i \in [start,end] \cdot p(i))$\\
m: $\mathbb{N} \times \mathbb{N} \rightarrow \mathbb{R}$\\
m($start,end) \equiv (+(i.\mathbb{N} | i \in [start,end] \cdot i \times
p(i)))/prb(start,end)$\\
mg() $\equiv +(i.\mathbb{N} | i \in [0,255] \cdot i \times p(i))$\\
sigma2b1: $\mathbb{N} \rightarrow \mathbb{R}$\\
sigma2b1($t1) \equiv prb(0,t1) \times (m(0,t1)-mg())^2 + prb(t1+1,255)
\times (m(t1+1,255)-mg())^2$\\
sigma2b2: $\mathbb{N} \times \mathbb{N} \rightarrow
\mathbb{R}$\\
sigma2b2($t1,t2) \equiv prb(0,t1) \times (m(0,t1)-mg())^2 + prb(t1+1,t2)
\times (m(t1+1,t2)-mg())^2 + prb(t2+1,255) \times (m(t2+1,255)-mg())^2$\\
\an{You didn't mention I should change here. But inspired by implementation, I
removed one parameter from every function, and got rid of 4 functions by reuse
existing ones. I think it's much better than before.}

\newpage

\section{MIS of Output Module} \label{Md_Output}

\subsection{Module}

Output

\subsection{Uses}

ImageData \ref{Md_ImageDS},
ThresCal in Section \ref{Md_Calculation}, ImageVerify in Section
\ref{Md_Verify}

\subsection{Syntax}

\subsubsection{Exported Constants}

\subsubsection{Exported Access Programs}

\begin{center}
\begin{tabular}{p{4cm} p{2cm} p{2cm} p{6cm}}
\hline
\textbf{Name} & \textbf{In} & \textbf{Out} & \textbf{Exceptions} \\
\hline
displayThresholds & - & - &\\
writeOutput & $s$: string & - &
skipSegmentation, badSegmentation, badDirectory \\
createSegmentation & - & - &\\
segImage & - & imageData &  \\
\hline
\end{tabular}
\end{center}

\subsection{Semantics}

\subsubsection{State Variables}

img: imageData\\
segImage: imageData\\
$j$: $\mathbb{N}$

\subsubsection{Environment Variables}

outputFile: a bitmap file

\subsubsection{Assumptions}

If this module is intended to be used on frame $j$, than it should be called
after using ThresCal in Section \ref{Md_Calculation} on frame $j$. The reason
isthat this module uses ThresCal.frameIndex to get current frame number.

\subsubsection{Access Routine Semantics}

\noindent Output.segImage:
\begin{itemize}
\item output: $out :=$ segImage
\item exception: none 
\end{itemize}

\noindent displayThresholds():
\begin{itemize}
\item transition:
This function has the following procedural specification:
\begin{enumerate}
\item If ThresCal.isThresValid(), continue.
\item $j$ := ThresCal.frameIndex
\item If ThresCal.chosenThresNum == 1, use Hardware-Hiding Module to display
thefollowing message:\\
``The single threshold value for frame $j$ is k= " + k1 + ``."
\item If ThresCal.chosenThresNum == 2, use Hardware-Hiding Module to display
thefollowing message:\\
``"The multiple threshold values for frame $j$ are k1= " + k1 +
``, k2= " + k2 + ``."
\end{enumerate}
\item output: none
\item exception: none
\end{itemize}

\noindent writeOutput($s$):
This method use segImage to write a outputFile to the environment using the
following procedural specification:\\

\# The first step is not necessary for the Control Module in this document,
which calls displayThresholds() that updates $j$, but it is necessary if
displayThresholds() is not called in advance.
\begin{enumerate}
\item $j$ := ThresCal.frameIndex
\item If ThresCal.isThresValid, continue.
\item createSegmentation()
\item if ImageVerify.verifyImageData(segImage) and
ImageVerify.compareSizes(img,segImage), continue.
\item Use data from segImage to write an outputFile to the environment
\end{enumerate}
\begin{itemize}
\item transition: none
\item output: none
\item exception: $exc := $\\
$\neg$ ThresCal.isThresValid $\implies$ skipSegmentation\\
$\neg$ (ImageVerify.verifyImageData(segImage) $\land$
ImageVerify.compareSizes(img, segImage)) $\implies$ badSegmentation\\
fail to write a file to the output directory
$\implies$ badDirectory\\
\end{itemize}

\noindent createSegmentation():
\begin{itemize}
\item transition:
This function modifies the state variables using the following procedural
specification:
\begin{enumerate}
\item img := Input.loadedImages[$j$]
\item Use local references $c, k1, k2$ for state variables in ThresCal.
\begin{itemize}
\item $c$ = ThresCal.chosenThresNum
\item k1 = ThresCal.ThresCal.k1
\item k2 = ThresCal.ThresCal.k2
\end{itemize}
\item pixelValue := $|| (pv: \mathbb{N} | pv \in$ img.pixelValue $\cdot (c =
1 \implies (pv > $k1$ \implies 255 |$ True $\implies 0) |\ c = 2 \implies (pv >
k2 \implies 255 |\ $k2$ \ge pv > $k1$ \implies
128 |$ True $\implies 0))$
    \item
    segImage := new ImageData(img.width,img.height,pixelValue)
\end{enumerate}
\item output: none
\item exception: none
\end{itemize}

\subsubsection{Local Functions}

None

\newpage

\section{MIS of Image Verification} \label{Md_Verify}

\subsection{Module}

ImageVerify

\subsection{Uses}

none

\subsection{Syntax}

\subsubsection{Exported Constants}

\subsubsection{Exported Access Programs}

\begin{center}
\begin{tabular}{p{3cm} p{3.5cm} p{2cm} p{6cm}}
\hline
\textbf{Name} & \textbf{In} & \textbf{Out} & \textbf{Exceptions} \\
\hline
verifyImageData & $img$: imageData & boolean & badSize,
badPixelData\\
compareSizes & $img1$: imageData,  $img2$: imageData & boolean & badSize2\\
\hline
\end{tabular}
\end{center}

\subsection{Semantics}

\subsubsection{State Variables}

None

\subsubsection{Environment Variables}

None

\subsubsection{Assumptions}

compareSizes($img1$, $img2$) does not check if these two inputs are
valid, it assumes that during the previous steps, the software has called
verifyImageData($img$) to verify these two inputs individually. \wss{This
accessprogram seems to be misnamed. The images are only being compared to verify
  that they are the same size.  I assumed that returning True would mean that
  the images were the same.}
\an{I agree. Changed the names.}

\subsubsection{Access Routine Semantics}

\noindent verifyImageData($img$):
The parameter $img$ is an instance of Image Data Structure.
\begin{itemize}
\item transition: none
\item output: := $img$.width $> 0 \land img$.height $> 0 \land
(\forall pv \in img.$pixelValue$ \cdot\ 0 \le pv \le 255)$
\item exception: $exc :=$\\
$img$.width $< 0 \lor img$.height $< 0 \implies$ badSize\\
$(\exists pv \in img.$pixelValue$ \cdot\ pv < 0 \lor pv > 255) \implies$
badPixelData
\end{itemize}

\noindent compareSizes($img1, img2$):
The parameters $img1$ and $img2$ are instances of Image Data Structure.
\begin{itemize}
\item transition: none
\item output: := ($img1$.width = $img2$.width) $\land (img1$.height =
$img2$.height)
\item exception: $exc :=$\\
$img1$.width $\not= img2$.width $\lor img1$.height $\not=
img2$.height$\implies$badSize2\\
\end{itemize}

\subsubsection{Local Functions}

None

\newpage

\bibliographystyle {plainnat}
\bibliography {../../../refs/References}

\newpage

\section{Appendix} \label{Appendix}

\renewcommand{\arraystretch}{1.2}

\begin{longtable}{l p{12cm}}
\caption{Possible Exceptions} \\
\toprule
\textbf{Message ID} & \textbf{Error and Warning Message} \\
\midrule
badInputFileName & Error: cannot find the file $s$.\\ badInputFileFormat &
Error: the format of file $s$ is not supported.\\
emptyFrameJ & Warning: frame $j$ is not loaded.\\
badWidthInput & Error: image width cannot be negative.\\
badHeightInput & Error: image height cannot be negative.\\
badPixelValueLength & Error: the length of image pixel value sequence must
equalto width $\times$ height.\\
skipCalculation & Warning: thresholds for frame $j$ are not calculated.\\
badResult & Error: incorrect thresholds calculation for frame $j$\\
badChoiceInput & Error: input is not a number from the set, please read the
following instructions carefully and try again:\\
skipSegmentation & Warning: frame $j$ is not segmented nor saved.\\
badSegmentation & Error: frame $j$ is not correctly segmented."\\
badDirectory & Error: Cannot write the output image to directory $s$\\
badSize & Error: invalid image size.\\
badPixelData & Error: One or more pixel values are outside the bound of [0,
255].\\
badSize2 & Error: inconsistent image sizes.\\
\bottomrule
\end{longtable}

\end{document}